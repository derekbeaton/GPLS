% !TeX program = pdfLaTeX
\documentclass[12pt]{article}
\usepackage{amsmath}
\usepackage{graphicx,psfrag,epsf}
\usepackage{enumerate}
\usepackage{natbib}
\usepackage{textcomp}
\usepackage[hyphens]{url} % not crucial - just used below for the URL
\usepackage{hyperref}
\providecommand{\tightlist}{%
  \setlength{\itemsep}{0pt}\setlength{\parskip}{0pt}}

%\pdfminorversion=4
% NOTE: To produce blinded version, replace "0" with "1" below.
\newcommand{\blind}{0}

% DON'T change margins - should be 1 inch all around.
\addtolength{\oddsidemargin}{-.5in}%
\addtolength{\evensidemargin}{-.5in}%
\addtolength{\textwidth}{1in}%
\addtolength{\textheight}{1.3in}%
\addtolength{\topmargin}{-.8in}%

%% load any required packages here




\usepackage{float}
\usepackage{bbold}
\usepackage{subfig}
\usepackage{graphicx}
\usepackage[utf8]{inputenc}
\usepackage[T1]{fontenc}
\usepackage{booktabs}
\usepackage{algorithm2e}
\usepackage{caption}
\usepackage{tabularx}
\usepackage{verbatim}
\usepackage{xcolor}

\begin{document}


\def\spacingset#1{\renewcommand{\baselinestretch}%
{#1}\small\normalsize} \spacingset{1}


%%%%%%%%%%%%%%%%%%%%%%%%%%%%%%%%%%%%%%%%%%%%%%%%%%%%%%%%%%%%%%%%%%%%%%%%%%%%%%

\if0\blind
{
  \title{\bf Supplemental Material for `A generalization of partial least squares
regression and correspondence analysis for categorical and mixed data:
An application with the ADNI data'}

  \author{
        Derek Beaton \\
    Rotman Research Institute, Baycrest Health Sciences\\
     and \\     ADNI \thanks{Data used in preparation of this article were obtained from the
Alzheimer's Disease Neuroimaging Initiative (ADNI) database
(\url{http://adni.loni.usc.edu/}). As such, the investigators within the
ADNI contributed to the design and implementation of ADNI and/or
provided data but did not participate in analysis or writing of this
report. A complete listing of ADNI investigators can be found at
http://adni.loni.ucla.edu/wpcontent/uploads/how\_to\_apply/ADNI\_Acknowledgement\_List.pdf} \\
    ADNI\\
     and \\     Gilbert Saporta \\
    Conservatoire National des Arts et Metiers\\
     and \\     Hervé Abdi \\
    Behavioral and Brain Sciences, The University of Texas at Dallas\\
      }
  \maketitle
} \fi

\if1\blind
{
  \bigskip
  \bigskip
  \bigskip
  \begin{center}
    {\LARGE\bf Supplemental Material for `A generalization of partial least squares
regression and correspondence analysis for categorical and mixed data:
An application with the ADNI data'}
  \end{center}
  \medskip
} \fi

\bigskip
\begin{abstract}
We provide supplemental material for partial least
squares-correspondence analysis-regression (PLS-CA-R\_ that highlights
how PLS-CA-R provides the baiss for generalization of numerous
cross-decomposition methods (e.g., PLS, CCA, RRR) as well as ridge-like
regularization. Here we provide additional material and explanations
(e.g., algorithms) based on the formulation in the main text.
\end{abstract}

\noindent%
{\it Keywords:} generalized partial least squares, canonical correlation analysis, reduced rank regression, ridge regularization, R package
\vfill

\newpage
\spacingset{1.45} % DON'T change the spacing!

\hypertarget{introduction}{%
\section{Introduction}\label{introduction}}

This Appendix provides additional details on PLS-CA-R and, in
particular, a variety of extensions of and generalizations from
PLS-CA-R. We also provide additional details on some concepts, such as
the generalized singular value decomposition. As established in the main
text, PLS-CA-R provides a generalization of PLS-R for categorical and
mixed data. However, PLS-CA-R provides the basis for futher
generalizations that extend to other optimizations, alternate metrics,
different PLS algorithms, and even ridge-like regularization. In this
Appendix we explain those additional generalizations and variations
based on how we established PLS-CA-R in the main text.

First, we explain the relationship between the SVD and GSVD in a more
detail than in the main text. Following that, we extend the concept of
the GSVD triplet for PLS with what we call ``the GPLSSVD sextuplet''.
Second, we show how the GPLSSVD triplet allows us to perform PLS-CA-R,
as well as other cross-decomposition techniques, more easily. From
there, we use the GPLSSVD triplet as a way to further simplify the three
primary PLS algorithms (regression, correlation, canonical). Third we
provide a short discussion on how the GPLSSVD---which is inspired from
PLS-CA-R---gives us a more unified way to accomodate different
optimizations (e.g., partial least squares vs.~canonical correlation)
and different weights or metrics. We then present two ways to perform a
ridge-like regularization with an emphasis on PLS-CA-R, but the
regularization applies to any technique under the GPLSSVD framework.
Finally, we point out numerous strategies and approaches for inference
and stability assessment that are easily adapted for PLS-CA-R
specifically and GPLSSVD generally.

\hypertarget{the-svd-gsvd-and-gplssvd}{%
\section{The SVD, GSVD, and GPLSSVD}\label{the-svd-gsvd-and-gplssvd}}

The SVD of \({\bf X}\) is \begin{equation}
{\bf X} = {\bf U}{\boldsymbol \Delta}{\bf V}^{T},
\end{equation} where
\({\bf U}^{T}{\bf U} = {\bf I} = {\bf V}^{T}{\bf V}\). The GSVD of
\({\bf X}\) is \begin{equation}
{\bf X} = {\bf P}{\boldsymbol \Delta}{\bf Q}^{T},
\end{equation} where
\({\bf P}^{T}{\bf M}{\bf P} = {\bf I} = {\bf Q}^{T}{\bf W}{\bf Q}\).
Practically, the GSVD is performed through the SVD as
\(\widetilde{\mathbf X} = {\mathbf M}^{\frac{1}{2}}{\mathbf X}{\mathbf W}^{\frac{1}{2}} = {\mathbf U} {\boldsymbol \Delta} {\mathbf V}^{T}\),
where the generalized singular vectors are computed from the singular
vectors as \({\mathbf P} = {\mathbf M}^{-\frac{1}{2}}{\mathbf U}\) and
\({\mathbf Q} = {\mathbf W}^{-\frac{1}{2}}{\mathbf V}\). The
relationship between the SVD and GSVD can be expressed through what we
decompose as \begin{equation}
\widetilde{\mathbf X} = {\mathbf M}^{\frac{1}{2}}{\mathbf X}{\mathbf W}^{\frac{1}{2}} \Longleftrightarrow {\mathbf X} = {\mathbf M}^{-\frac{1}{2}}\widetilde{\mathbf X}{\mathbf W}^{-\frac{1}{2}}.
\end{equation} As noted in the main text, the GSVD can be presented in
``triplet notation'' as
\(\mathrm{GSVD(}{\mathbf M}, {\mathbf X}, {\mathbf W}\mathrm{)}\).

\hypertarget{from-gsvd-triplet-to-gplssvd-sextuplet}{%
\subsection{From GSVD triplet to GPLSSVD
sextuplet}\label{from-gsvd-triplet-to-gplssvd-sextuplet}}

We introduce an extension of the GSVD triplet for PLS, called the
``GPLSSVD sextuplet''. The GPLSSVD sextuplet helps us in two ways: (1)
it simplifies some of the notation and decomposition concepts, and (2)
it provides the basis for generalization of cross-decomposition methods
(e.g., canonical correlation, reduced rank regression).

The ``GPLSSVD sextuplet'' takes the form of
\(\mathrm{GPLSSVD(} {\mathbf M}_{\mathbf X}, {\mathbf Z}_{\mathbf X}, {\mathbf W}_{\mathbf X}, {\mathbf M}_{\mathbf Y}, {\mathbf Z}_{\mathbf Y}, {\mathbf W}_{\mathbf Y} \mathrm{)}\)
and like the GSVD, decomposes a matrix
\({\mathbf Z}_{\mathbf R} = ({\mathbf M}_{\mathbf X}^{\frac{1}{2}}{\mathbf Z}_{\mathbf X})^{T} {\mathbf M}_{\mathbf Y}^{\frac{1}{2}} {\mathbf Z}_{\mathbf Y}\)
as

\begin{equation}
{\mathbf Z}_{\mathbf R} = {\mathbf P} {\boldsymbol \Delta} {\mathbf Q}^{T}
\textrm{ with }
{\mathbf P}^{T}{\mathbf W}_{\mathbf X}{\mathbf P} = {\mathbf I} =
{\mathbf Q}^{T}{\mathbf W}_{\mathbf Y}{\mathbf Q}.
\end{equation} From this decomposition, the GPLSSVD produces latent
variables as \begin{equation}
{\mathbf L}_{\mathbf X} 
= {\mathbf Z}_{\mathbf X}{\mathbf W}_{\mathbf X}{\mathbf P} 
\textrm{ and } 
{\mathbf L}_{\mathbf Y} = 
{\mathbf Z}_{\mathbf Y}{\mathbf W}_{\mathbf Y}{\mathbf Q}
\textrm{ where }
{\mathbf L}_{\mathbf X}^{T} {\mathbf L}_{\mathbf Y} 
= {\boldsymbol \Delta}. 
\end{equation}

Alternatively, we can show the GPLSSVD through the SVD. Let us refer to
\({\widetilde{\mathbf Z}_{\mathbf X}} = {\mathbf M}_{\mathbf X}^{\frac{1}{2}}{\mathbf Z}_{\mathbf X}{\mathbf W}_{\mathbf X}^{\frac{1}{2}}\)
and
\({\widetilde{\mathbf Z}_{\mathbf Y}} = {\mathbf M}_{\mathbf Y}^{\frac{1}{2}}{\mathbf Z}_{\mathbf Y}{\mathbf W}_{\mathbf Y}^{\frac{1}{2}}\),
where
\(\widetilde{\mathbf Z}_{\mathbf R} = {\widetilde{\mathbf Z}_{\mathbf X}}^{T}{\widetilde{\mathbf Z}_{\mathbf Y}} = ({\mathbf M}_{\mathbf X}^{\frac{1}{2}}{\mathbf Z}_{\mathbf X}{\mathbf W}_{\mathbf X}^{\frac{1}{2}})^{T}({\mathbf M}_{\mathbf Y}^{\frac{1}{2}}{\mathbf Z}_{\mathbf Y}{\mathbf W}_{\mathbf Y}^{\frac{1}{2}})\).
We decompose \(\widetilde{\mathbf Z}_{\mathbf R}\) as

\begin{equation}
\widetilde{\mathbf Z}_{\mathbf R} = {\mathbf U} {\boldsymbol \Delta} {\mathbf V}^{T}
\textrm{ with }
{\mathbf U}^{T}{\mathbf U} = {\mathbf I} =
{\mathbf V}^{T}{\mathbf V}.
\end{equation} We can also compute the latent variables as
\begin{equation}
{\mathbf L}_{\mathbf X} 
= \widetilde{\mathbf Z}_{\mathbf X}{\mathbf U} 
\textrm{ and } 
{\mathbf L}_{\mathbf Y} = 
\widetilde{\mathbf Z}_{\mathbf Y}{\mathbf V}
\textrm{ where }
{\mathbf L}_{\mathbf X}^{T} {\mathbf L}_{\mathbf Y} 
= {\boldsymbol \Delta}. 
\end{equation}

Like with the SVD and GSVD, the GPLSSVD has the same orthogonality
constraints:
\({\mathbf U}^{T}{\mathbf U} = {\mathbf I} = {\mathbf V}^{T}{\mathbf V}\),
or
\({\mathbf P}^{T}{\mathbf M}_{\mathbf Y}{\mathbf P} = {\mathbf I} = {\mathbf Q}^{T}{\mathbf W}_{\mathbf Y}{\mathbf Q}\).
The relationship between the singular vectors are
\({\mathbf U} = {\mathbf W}_{\bf X}^{\frac{1}{2}}{\mathbf P} \Longleftrightarrow {\mathbf P} = {\mathbf M}_{\bf X}^{-\frac{1}{2}}{\mathbf U}\)
and
\({\mathbf V} = {\mathbf W}_{\bf Y}^{\frac{1}{2}}{\mathbf Q} \Longleftrightarrow {\mathbf Q} = {\mathbf W}_{\bf Y}^{-\frac{1}{2}}{\mathbf V}\).
We can also compute component (a.k.a. factor) scores from the GPLSSVD as
\({\bf F}_{J} = {\mathbf W}_{\bf X}{\mathbf P}{\boldsymbol \Delta}\) and
\({\bf F}_{K} = {\mathbf W}_{\bf Y}{\mathbf Q}{\boldsymbol \Delta}\). An
alternate form of the component scores are
\({\bf F}_{J}' = {\mathbf W}_{\bf X}{\mathbf P}\) and
\({\bf F}_{K}' = {\mathbf W}_{\bf Y}{\mathbf Q}\).

Finally, we note one last extension to the GPLSSVD to make it the
``GPLSSVD septuplet'' (similarly we also a ``GSVD quadruplet''). We can
include desired rank to return as an input parameter for the GPLSSVD
septuplet (and GSVD quadruplet) as
\(\mathrm{GPLSSVD(} {\mathbf M}_{\mathbf X}, {\mathbf Z}_{\mathbf X}, {\mathbf W}_{\mathbf X}, {\mathbf M}_{\mathbf Y}, {\mathbf Z}_{\mathbf Y}, {\mathbf W}_{\mathbf Y}, C \mathrm{)}\)
(and
\(\mathrm{GSVD(} {\mathbf M}_{\mathbf X}, {\mathbf Z}_{\mathbf X}, {\mathbf W}_{\mathbf X}, C \mathrm{)}\)),
where \(C\) is an integer to indicate the rank of the solution desired.
For example, when \(C = 1\) the GPLSSVD (and GSVD) return a rank 1
solution. The \(C\) parameter could be any value (constrained to the
minimum rank of either \({\bf X}\) or \({\bf Y}\)), but \(C=1\) is
particlarly convenient for our following generalizations.

\hypertarget{pls-and-gpls-algorithms}{%
\section{PLS and GPLS algorithms}\label{pls-and-gpls-algorithms}}

Though we have presented PLS-CA regression as a generalization of PLS
regression that accomodates virutally any data type (by way of CA), the
way we formalized PLS-CA regression leads to further variants and
broader generalizations. These generalizations span (1) various PLS, CA,
and related approaches, (2) several typical PLS algorithms, (3) a
variety of optimizations (e.g., canonical correlation), and (4)
ridge-like regularization.

There exist three commonly used PLS algorithms: (1) PLS regression
(PLS-REG) decomposition
\citep{wold1975soft, wold_collinearity_1984, wold_pls-regression_2001, abdi_partial_2010-1},
(2) PLS correlation (PLS-COR) decomposition
\citep{bookstein1994partial, ketterlinus1989partial} generally more
known in neuroimaging
\citep{mcintosh_spatial_1996, mcintosh_partial_2004, krishnan_partial_2011}
and also has numerous alternate names such as PLS-SVD, co-inertia
\citep[\citet{dray2014}]{doledec1994}, and Tucker's interbattery factor
analysis \citep{tucker_inter-battery_1958} amongst others \citep[see
also][]{beaton_partial_2016}, and (3) PLS canonical (PLS-CAN)
decomposition \citep{tenenhaus_regression_1998, wegelin2000survey} which
is a symmetric method (like PLS-COR) with iterative deflation (like
PLS-REG).

Based on how we formalized PLS-CA regression, we now show how PLS-CA
regression provides the basis of generalizations of these three
algorithms, as well as further optimizations, similar to
\citet{borga_unified_1992}, \citet{indahl2009canonical}, and
\citet{de2019pls}. But we do so in a more comprehensive way that
incorporates more methods than other unification strategies, and we also
do so in a way that accomodates multiple data types. We refer to the
generalization of the three previously mentioned PLS techniques under
the umbrella of generalized partial least squares (GPLS) as GPLS-COR,
GPLS-REG, and GPLS-CAN, for the ``correlation'', ``regression'', and
``canonical'' decompositions respectively. GPLS-COR and GPLS-CAN are
symmetric decomposition approaches where neither \({\mathbf X}\) nor
\({\mathbf Y}\) are privileged. GPLS-REG is an asymmetric decomposition
approach where \({\mathbf X}\) is privileged. We present the GPLS-COR,
GPLS-REG, and then GPLS-CAN algorithms with their respective
optimizations. We do so in the previously mentioned order because
GPLS-COR---by way of the GPLSSVD---is used as the basis of all three
algorithms and GPLS-CAN shares features and concepts with both GPLS-COR
and GPLS-REG. For all of these we rely on the formlization of PLS-CA
regression as established in the main text.

\hypertarget{gpls-cor}{%
\subsection{GPLS-COR}\label{gpls-cor}}

The GPLS-COR decomposition is the simplest GPLS technique. It requires
only a single pass of the SVD---or in our case the GPLSSVD. There are no
explicit iterative steps in GPLS-COR. GPLS-COR takes as input the two
preprocessed matrices---\({\mathbf Z}_{\mathbf X}\) and
\({\mathbf Z}_{\mathbf Y}\)---and their respective row and column
weights: \({\mathbf M}_{\mathbf X}\) and \({\mathbf W}_{\mathbf X}\) for
\({\mathbf Z}_{\mathbf X}\), and \({\mathbf M}_{\mathbf Y}\) and
\({\mathbf W}_{\mathbf Y}\) for \({\mathbf Z}_{\mathbf Y}\), where \(C\)
is the desired number of components to return. GPLS-COR is shown in
Algorithm \ref{algo:plsc}.

\RestyleAlgo{boxed}
\begin{algorithm}
\DontPrintSemicolon
\SetAlgoLined
\KwResult{Generalized PLS-correlation between ${\mathbf Z}_{{\mathbf X}}$ and ${\mathbf Z}_{{\mathbf Y}}$}
\SetKwInOut{Input}{Input}\SetKwInOut{Output}{Output}
\Input{${\mathbf M}_{\mathbf{X}}$, ${\mathbf Z}_{{\mathbf X}}$, ${\mathbf W}_{\mathbf{X}}$, ${\mathbf M}_{\mathbf{Y}}$, ${\mathbf Z}_{{\mathbf Y}}$, ${\mathbf W}_{\mathbf{Y}}$, $C$}
\Output{${\boldsymbol \Delta}$, ${\mathbf U}$, ${\mathbf V}$, ${\mathbf P}$, ${\mathbf Q}$, ${\mathbf F}_{J}$, ${\mathbf F}_{K}$, ${\mathbf L}_{{\mathbf X}}$, ${\mathbf L}_{{\mathbf Y}}$}
\BlankLine
  $\mathrm{GPLSSVD(} {\mathbf M}_{\mathbf{X}}, {\mathbf Z}_{{\mathbf X}}, {\mathbf W}_{\mathbf{X}}, {\mathbf M}_{\mathbf{Y}}, {\mathbf Z}_{{\mathbf Y}}, {\mathbf W}_{\mathbf{Y}}, C \mathrm{)}$ \\
\caption{Generalized PLS-correlation algorithm. GPLS-COR is the GPLSSVD and provides the basis of other GPLS techniques. Furthermore, GPLS-COR easily allows for a variety of optmizations for examples canonical correlation, reduced rank regression (redundancy analysis), and even ridge-like regularization, which then extend to the other GPLS algorithms (i.e., regression and canonical decompositions). Note that this is a truncated version of the algorithm and does not include all of the GPLSSVD outputs.}
\label{algo:plsc}
\end{algorithm}

GPLS-COR maximizes the relationship between \({\mathbf L}_{\mathbf X}\)
and \({\mathbf L}_{\mathbf Y}\) with the orthogonality constraint
\({\boldsymbol \ell}_{{\mathbf X},c}^{T}{\boldsymbol \ell}_{{\mathbf Y},c'} = 0\)
when \(c \neq c'\) where
\({\boldsymbol \ell}_{{\mathbf X},c}^{T}{\boldsymbol \ell}_{{\mathbf Y},c} = \delta_{c}\)
and thus
\({\mathbf L}_{\mathbf X}^{T}{\mathbf L}_{\mathbf Y} = {\mathbf U}^{T}\widetilde{\mathbf Z}_{\mathbf X}^{T}\widetilde{\mathbf Z}_{\mathbf Y}{\mathbf V}^{T} = {\mathbf U}^{T}\widetilde{\mathbf Z}_{\mathbf R}{\mathbf V}^{T} = {\mathbf U}^{T}{\mathbf U}{\boldsymbol \Delta}{\mathbf V}^{T}{\mathbf V}^{T} = {\boldsymbol \Delta}\).
We can also show this with the generalized vectors and constraints as
\({\mathbf L}_{\mathbf X}^{T}{\mathbf L}_{\mathbf Y} = {\mathbf P}^{T}{\mathbf W}_{\mathbf X}{\mathbf Z}_{\mathbf X}^{T}{\mathbf M}_{\mathbf X}^{\frac{1}{2}}{\mathbf M}_{\mathbf Y}^{\frac{1}{2}}{\mathbf Z}_{\mathbf Y}{\mathbf W}_{\mathbf Y}{\mathbf Q}^{T} = {\mathbf P}^{T}{\mathbf W}_{\mathbf X}{\mathbf P}{\boldsymbol \Delta}{\mathbf Q}^{T}{\mathbf W}_{\mathbf Y}{\mathbf Q} = {\boldsymbol \Delta}\).

The primary example of GPLS-COR is the standard ``PLS correlation''
approach. Let's assume that \({\mathbf X}\) and \({\mathbf Y}\) are
comprised of continuous data. So, \({\mathbf Z}_{\mathbf X}\) and
\({\mathbf Z}_{\mathbf Y}\) are column-wise centered (and/or normalized)
versions of \({\mathbf X}\) and \({\mathbf Y}\). When all weight
matrices are identity matrices, then the GPLSSVD implements the ``PLS
correlation'' decomposition as
\(\mathrm{GPLSSVD(} {\mathbf I}, {\mathbf Z}_{\mathbf X}, {\mathbf I}, {\mathbf I}, {\mathbf Z}_{\mathbf Y}, {\mathbf I} \mathrm{)}\)
and also as shown in Algorithm \ref{algo:plsc}. However, the way we
established the GPLSSVD and Algorithm \ref{algo:plsc} allows us to
obtain the results of three of the most common cross-decomposition
(``two-table'') techniques: PLS correlation (shown above), reduced rank
regression (RRR, a.k.a., reduced rank regression {[}RDA{]}), and
canonical correlation analysis (CCA). RRR/RDA is performed as
\(\mathrm{GPLSSVD(} {\mathbf I}, {\mathbf Z}_{{\mathbf X}}, ({\mathbf Z}_{\mathbf X}^{T}{\mathbf Z}_{\mathbf X})^{-1}, {\mathbf I}, {\mathbf Z}_{{\mathbf Y}}, {\mathbf I}\mathrm{)}\)
and CCA is performed as
\(\mathrm{GPLSSVD(} {\mathbf I}, {\mathbf Z}_{{\mathbf X}}, ({\mathbf Z}_{\mathbf X}^{T}{\mathbf Z}_{\mathbf X})^{-1}, {\mathbf I}, {\mathbf Z}_{{\mathbf Y}}, ({\mathbf Z}_{\mathbf Y}^{T}{\mathbf Z}_{\mathbf Y})^{-1}\mathrm{)}\).
In Table \ref{table:crossdecomp}, we show how these three techniques are
related through the GPLSSVD by highlighting that they are merely a
change in choice of weights (metrics).

\begin{table}
\centering
\begin{tabular}{ l  c  c  c  c }
  Method & ${\bf M}_{\bf X}$ & ${\bf W}_{\bf X}$ & ${\bf M}_{\bf Y}$ & ${\bf W}_{\bf Y}$ \\
  \hline            
  PLS-COR & ${\bf I}$ & ${\bf I}$ & ${\bf I}$ & ${\bf I}$ \\
  RRR/RDA & ${\bf I}$ & $({\mathbf Z}_{\mathbf X}^{T}{\mathbf Z}_{\mathbf X})^{-1}$ & ${\bf I}$ & ${\bf I}$ \\
  CCA & ${\bf I}$ & $({\mathbf Z}_{\mathbf X}^{T}{\mathbf Z}_{\mathbf X})^{-1}$ & ${\bf I}$ & $({\mathbf Z}_{\mathbf Y}^{T}{\mathbf Z}_{\mathbf Y})^{-1}$ \\
  \hline  
\end{tabular}
\caption{The primary cross-decomposition techniques framed as GPLSSVD approaches.}\label{table:crossdecomp}
\end{table}

Furthermore, these three variants---PLSC, CCA, and RDA/RRR---also
generalize discriminant analyses under different optimizations so long
either \({\mathbf X}\) or \({\mathbf Y}\) (depending on the technique)
is a dummy-coded (complete disjunctive) matrix, where each observation
(row) is assigned to a specific group or category (columns). Going
further, if we have disjunctive (or pseudo-disjunctive) data---as we do
in the main text--- we can use GPLSSVD to obtain the results of PLS-CA
``correlation'' \citep{beaton_partial_2016}. Using the same matrices as
established in the main text, PLS-CA ``correlation'' is
\(\mathrm{GPLSSVD(} {\mathbf M}_{\bf X}^{-1}, {\mathbf Z}_{{\mathbf X}}, {\mathbf W}_{\bf X}^{-1}, {\mathbf M}_{\bf Y}^{-1}, {\mathbf Z}_{{\mathbf Y}}, {\mathbf W}_{\bf Y}^{-1}\mathrm{)}\),
where \({\mathbf M}_{\bf X}\) and \({\mathbf M}_{\bf Y}\) are diagonal
matrices of row frequencies for each matrix, where
\({\mathbf W}_{\bf X}\) and \({\mathbf W}_{\bf Y}\) are diagonal
matrices of column frequencies for each matrix, and
\({\mathbf Z}_{{\mathbf X}}\) and \({\mathbf Z}_{{\mathbf Y}}\) are the
deviations from independence matrices.

GPLS-COR---which is just the GPLSSVD---provides the basis for the other
two algorithms: both GPLS-REG and GPLS-CAN make use of GPLS-COR (i.e.,
the GPLSSVD) with rank 1 solutions iteratively.

\hypertarget{gpls-reg}{%
\subsection{GPLS-REG}\label{gpls-reg}}

The GPLS-REG decomposition uses the GPLSSVD septuplet iteratively for
\(C\) iterations, with only a rank 1 solution is provided for each use
of the GPLSSVD. Alternatively, GPLS-REG can be thought of a direct
extension of GPLS-COR as defined in Algorithm \ref{algo:plsc}.

Then the two data matrices---\({\mathbf Z}_{\mathbf X}\) and
\({\mathbf Z}_{\mathbf Y}\)---are deflated for each step asymmetrically,
with a privileged \({\mathbf Z}_{\mathbf X}\). GPLS-REG is shown in
Algorithm \ref{algo:plscar}.

\RestyleAlgo{boxed}
\begin{algorithm}
\DontPrintSemicolon
\SetAlgoLined
\KwResult{Generalized PLS-regression between ${\mathbf Z}_{{\mathbf X}}$ and ${\mathbf Z}_{{\mathbf Y}}$}
\SetKwInOut{Input}{Input}\SetKwInOut{Output}{Output}
\Input{${\mathbf M}_{\mathbf{X}}$, ${\mathbf Z}_{{\mathbf X}}$, ${\mathbf W}_{\mathbf{X}}$, ${\mathbf M}_{\mathbf{Y}}$, ${\mathbf Z}_{{\mathbf Y}}$, ${\mathbf W}_{\mathbf{Y}}$, $C$}
\Output{$\widetilde{\boldsymbol \Delta}$, $\widetilde{\mathbf U}$, $\widetilde{\mathbf V}$, $\widetilde{\mathbf P}$, $\widetilde{\mathbf Q}$, $\widetilde{\mathbf F}_{J}$, $\widetilde{\mathbf F}_{K}$, ${\mathbf L}_{{\mathbf X}}$, ${\mathbf L}_{{\mathbf Y}}$, ${\mathbf T}_{{\mathbf X}}$, $\widehat{\mathbf U}$, ${\mathbf B}$}
\BlankLine
\For{$c=1, \dots, C$}{

  $\mathrm{GPLSSVD(} {\mathbf M}_{\mathbf{X}}, {\mathbf Z}_{{\mathbf X}}, {\mathbf W}_{\mathbf{X}}, {\mathbf M}_{\mathbf{Y}}, {\mathbf Z}_{{\mathbf Y}}, {\mathbf W}_{\mathbf{Y}}, 1 \mathrm{)}$ \\
  ${\mathbf t}_{\mathbf X} \leftarrow {\boldsymbol \ell}_{\mathbf X} \times {{\lvert\lvert {\boldsymbol \ell}_{\mathbf X} \rvert\rvert}^{-1}}$\\
  $b \leftarrow {\boldsymbol \ell}_{\mathbf Y}^{T}{\mathbf t}_{\mathbf X}$\\
  $\widehat{\mathbf u} \leftarrow ({\mathbf M}_{\mathbf X}^{\frac{1}{2}}{\mathbf Z}_{\mathbf X}{\mathbf W}_{\mathbf X}^{\frac{1}{2}})^{T}{\mathbf t}_{\mathbf X}$\\
  ${\mathbf Z}_{{\mathbf X}} \leftarrow {\mathbf Z}_{{\mathbf X}} - [{\mathbf M}_{\mathbf X}^{-\frac{1}{2}}({\mathbf t}_{\mathbf X}\widehat{\mathbf u}^{T}){\mathbf W}_{\mathbf X}^{-\frac{1}{2}}]$\\
  ${\mathbf Z}_{{\mathbf Y}} \leftarrow {\mathbf Z}_{{\mathbf Y}} - [{\mathbf M}_{\mathbf Y}^{-\frac{1}{2}}(b{\mathbf t}_{\mathbf X}{\mathbf{v}}^{T}){\mathbf W}_{\mathbf Y}^{-\frac{1}{2}}]$
}
\caption{Generalized PLS-regression algorithm. The results of a rank 1 GPLSSVD are used to compute the latent variables and values necessary for deflation of ${\mathbf Z}_{{\mathbf X}}$ and ${\mathbf Z}_{{\mathbf Y}}$. Note that this is a truncated version of the algorithm and does not include all of the GPLSSVD outputs.}
\label{algo:plscar}
\end{algorithm}

GPLS-REG maximizes the relationship between \({\mathbf L}_{\mathbf X}\)
and \({\mathbf L}_{\mathbf Y}\) with the orthogonality constraint
\({\boldsymbol \ell}_{{\mathbf X},c}^{T}{\boldsymbol \ell}_{{\mathbf X},c'} = 0\)
when \(c \neq c'\) where
\({\boldsymbol \ell}_{{\mathbf X},c}^{T}{\boldsymbol \ell}_{{\mathbf Y},c} = \delta_{c}\)
which is also
\(\mathrm{diag\{}{\mathbf L}_{\mathbf X}^{T}{\mathbf L}_{\mathbf Y}\mathrm{\}} = \mathrm{diag\{}\widetilde{\boldsymbol \Delta}\mathrm{\}}\).
When \({\mathbf Z}_{\mathbf X}\) and \({\mathbf Z}_{\mathbf Y}\) are
column-wise centered (and/or normalized) versions of \({\mathbf X}\) and
\({\mathbf Y}\), and when all weight matrices are identity matrices,
then GPLS-REG is (one of the) traditional ``PLS regression''
decomposition(s; akin to the SIMPLS algorithm in
\citet{tenenhaus_regression_1998}). We detail the regression
decomposition for categorical and mixed data in the main text.

\hypertarget{gpls-can}{%
\subsection{GPLS-CAN}\label{gpls-can}}

The GPLS-CAN decomposition can be thought of as a compromise between
GPLS-COR and GPLS-REG, it: (1) is symmetric like GPLS-COR, and (2) uses
the GPLSSVD septuplet iteratively for \(C\) iterations---with only a
rank 1 solution is provided for each use of the GPLSSVD---like GPLS-REG.
In GPLS-CAN, the two data matrices---\({\mathbf Z}_{\mathbf X}\) and
\({\mathbf Z}_{\mathbf Y}\)---are deflated for each iteration. GPLS-CAN
is shown in Algorithm \ref{algo:plscacan}

\RestyleAlgo{boxed}
\begin{algorithm}
\DontPrintSemicolon
\SetAlgoLined
\KwResult{Generalized PLS-canonical between ${\mathbf Z}_{{\mathbf X}}$ and ${\mathbf Z}_{{\mathbf Y}}$}
\SetKwInOut{Input}{Input}\SetKwInOut{Output}{Output}
\Input{${\mathbf M}_{\mathbf{X}}$, ${\mathbf Z}_{{\mathbf X}}$, ${\mathbf W}_{\mathbf{X}}$, ${\mathbf M}_{\mathbf{Y}}$, ${\mathbf Z}_{{\mathbf Y}}$, ${\mathbf W}_{\mathbf{Y}}$, $C$}
\Output{$\widetilde{\mathbf U}$, $\widetilde{\mathbf V}$, $\widetilde{\mathbf P}$, $\widetilde{\mathbf Q}$, $\widetilde{\mathbf F}_{J}$, $\widetilde{\mathbf F}_{K}$, ${\mathbf L}_{{\mathbf X}}$, ${\mathbf L}_{{\mathbf Y}}$, $\widetilde{\boldsymbol \Delta}$, ${\mathbf T}_{{\mathbf X}}$, ${\mathbf T}_{{\mathbf Y}}$, $\widehat{\mathbf U}$, $\widehat{\mathbf V}$}
\BlankLine
\For{$c=1, \dots, C$}{

  $\mathrm{GPLSSVD(} {\mathbf M}_{\mathbf{X}}, {\mathbf Z}_{{\mathbf X}}, {\mathbf W}_{\mathbf{X}}, {\mathbf M}_{\mathbf{Y}}, {\mathbf Z}_{{\mathbf Y}}, {\mathbf W}_{\mathbf{Y}}, 1 \mathrm{)}$ \\
  ${\mathbf t}_{\mathbf X} \leftarrow {\boldsymbol \ell}_{\mathbf X} \times {{\lvert\lvert {\boldsymbol \ell}_{\mathbf X} \rvert\rvert}^{-1}}$\\
  ${\mathbf t}_{\mathbf Y} \leftarrow {\boldsymbol \ell}_{\mathbf Y} \times {{\lvert\lvert {\boldsymbol \ell}_{\mathbf Y} \rvert\rvert}^{-1}}$\\
  $\widehat{\mathbf u} \leftarrow ({\mathbf M}_{\mathbf X}^{\frac{1}{2}}{\mathbf Z}_{\mathbf X}{\mathbf W}_{\mathbf X}^{\frac{1}{2}})^{T}{\mathbf t}_{\mathbf X}$\\
  $\widehat{\mathbf v} \leftarrow ({\mathbf M}_{\mathbf Y}^{\frac{1}{2}}{\mathbf Z}_{\mathbf Y}{\mathbf W}_{\mathbf Y}^{\frac{1}{2}})^{T}{\mathbf t}_{\mathbf Y}$\\  
  
  ${\mathbf Z}_{{\mathbf X}} \leftarrow {\mathbf Z}_{{\mathbf X}} - [{\mathbf M}_{\mathbf X}^{-\frac{1}{2}}({\mathbf t}_{\mathbf X}\widehat{\mathbf u}^{T}){\mathbf W}_{\mathbf X}^{-\frac{1}{2}}]$\\
   ${\mathbf Z}_{{\mathbf Y}} \leftarrow {\mathbf Z}_{{\mathbf Y}} - [{\mathbf M}_{\mathbf Y}^{-\frac{1}{2}}({\mathbf t}_{\mathbf Y}\widehat{\mathbf v}^{T}){\mathbf W}_{\mathbf Y}^{-\frac{1}{2}}]$
}
\caption{Generalized PLS-canonical algorithm. The results of a rank 1 GPLSSVD are used to compute the latent variables and values necessary for deflation of ${\mathbf Z}_{{\mathbf X}}$ and ${\mathbf Z}_{{\mathbf Y}}$. The deflation in GPLS-CAN is the same as in GPLS-REG in Algorithm \ref{algo:plscar}, but for both matrices. Note that this is a truncated version of the algorithm and does not include all of the GPLSSVD outputs.}
\label{algo:plscacan}
\end{algorithm}

GPLS-CAN maximizes the relationship between \({\mathbf L}_{\mathbf X}\)
and \({\mathbf L}_{\mathbf Y}\) with the orthogonality constraints
\({\boldsymbol \ell}_{{\mathbf X},c}^{T}{\boldsymbol \ell}_{{\mathbf X},c'} = 0\)
and
\({\boldsymbol \ell}_{{\mathbf Y},c}^{T}{\boldsymbol \ell}_{{\mathbf Y},c'} = 0\)
when \(c \neq c'\) where
\({\boldsymbol \ell}_{{\mathbf X},c}^{T}{\boldsymbol \ell}_{{\mathbf Y},c} = \delta_{c}\)
which is also
\(\mathrm{diag\{}{\mathbf L}_{\mathbf X}^{T}{\mathbf L}_{\mathbf Y}\mathrm{\}} = \mathrm{diag\{}\widetilde{\boldsymbol \Delta}\mathrm{\}}\).

\hypertarget{gpls-algorithms-summary}{%
\subsection{GPLS algorithms summary}\label{gpls-algorithms-summary}}

Note that across the three algorithms defined here, that the first
component is identical when the same preprocessed data and weights
(a.k.a. metrics) are provided to the GPLSSVD. In many cases, subsequent
components across the three algorithms differ, but generally do not
differ substantially. The similarities are because of the common
approach to maximization:
\({\boldsymbol \ell}_{{\mathbf X},c}^{T}{\boldsymbol \ell}_{{\mathbf Y},c} = \delta_{c}\).
The differences are because of the different orthogonality constraints
when \(c \neq c'\) where: (1) GPLS-COR in Algorithm \ref{algo:plsc} is
\({\boldsymbol \ell}_{{\mathbf X},c}^{T}{\boldsymbol \ell}_{{\mathbf Y},c'} = 0\),
(2) GPLS-REG in Algorithm \ref{algo:plscar} is
\({\boldsymbol \ell}_{{\mathbf X},c}^{T}{\boldsymbol \ell}_{{\mathbf X},c'} = 0\),
and (3) GPLS-CAN in Algorithm \ref{algo:plscacan} is both
\({\boldsymbol \ell}_{{\mathbf X},c}^{T}{\boldsymbol \ell}_{{\mathbf X},c'} = 0\)
and
\({\boldsymbol \ell}_{{\mathbf Y},c}^{T}{\boldsymbol \ell}_{{\mathbf Y},c'} = 0\).

\hypertarget{gpls-optimizations-and-further-generalizations}{%
\subsection{GPLS optimizations and further
generalizations}\label{gpls-optimizations-and-further-generalizations}}

From the GPLS perspective, we can better unify the wide variety of
approaches with similar goals but variations of metrics,
transformations, and optimizations that often appear under a wide
variety of names (e.g., PLS, CCA, interbattery factor analysis,
co-inertia analysis, canonical variates, PLS-CA, and so on; see
\citet{abdi2017canonical}). The way we defined the GPLS algorithms---in
particular with the weights applied to the rows and columns of each data
matrix---leads to numerous further generalizations.

Given the way we establish the GPLS algorithms here, we provide the
basis for further generalizations of many approaches. This is especially
true for the numerous variants of correspondecne analysis, such as power
transformations for CA \citep{greenacre2009power} alternate distance
metrics such as Hellinger distances
\citep{rao1995review, escofier1978analyse}, or ``non-symmetrical CA''
\citep{d1992non, kroonenberg1999nonsymmetric, takane1991relationships}.
For many of these approaches, the weight matrices---either for the rows
or columns of a given matrix---are what change between techniques. In
some of the aforementioned cases (e.g., power transformations) there are
also additional steps to preprocess the data.

\hypertarget{ridge-like-regularizations}{%
\subsection{Ridge-like
regularizations}\label{ridge-like-regularizations}}

We show two possible strategies for ridge-like regularization for the
GPLSSVD (which then applies to any of the algorithms we outline above).
We first show these two regularization approaches specifically for the
PLS-CA framework. From there we briefly discuss how these regularization
approaches extend to other techniques (e.g., CCA, PLS-COR) under the
GPLSSVD framework.

The first approach is based on Takane's regularized multiple CA
\citep{takane_regularized_2006} and regularized nonsymmetric CA
\citep{takane_regularized_2009-1}. To do so, it is convenient to
slightly reformulate PLS-CA-R, but we still require \({\mathbf X}\),
\({\mathbf Y}\), \({\mathbf O}_{\mathbf X}\),
\({\mathbf O}_{\mathbf Y}\), \({\mathbf E}_{\mathbf X}\), and
\({\mathbf E}_{\mathbf Y}\) as defined in the main text. First we
re-define
\({\mathbf Z}_{\mathbf X} = ({\mathbf O}_{\mathbf X} - {\mathbf E}_{\mathbf X}) \times (\mathbf{1}^{T}{\mathbf X1})\)
and
\({\mathbf Z}_{\mathbf Y} = ({\mathbf O}_{\mathbf Y} - {\mathbf E}_{\mathbf Y}) \times (\mathbf{1}^{T}{\mathbf Y1})\).
Next we define the following additional matrices:
\({\mathbf D}_{{\mathbf X},I} = \mathrm{diag\{ \mathbf{X1} \}}\), and
\({\mathbf D}_{{\mathbf Y},I} = \mathrm{diag\{ \mathbf{Y1} \}}\) which
are diagonal matrices of the row sums of \({\mathbf X}\) and
\({\mathbf Y}\), and
\({\mathbf D}_{{\mathbf X},J} = \mathrm{diag\{ \mathbf{1}^{T} \mathbf{X} \}}\),
and
\({\mathbf D}_{{\mathbf Y},K} = \mathrm{diag\{ \mathbf{1}^{T}\mathbf{Y} \}}\)
which are the column sums of \({\mathbf X}\) and \({\mathbf Y}\). Then
PLS-CA correlation, regression, and canonical decompositions replace the
GPLSSVD step in Algorithms \ref{algo:plsc}, \ref{algo:plscar},
\ref{algo:plscacan} with
\(\mathrm{GPLSSVD(}{\mathbf D}_{{\mathbf X},I}^{-1},{\mathbf Z}_{\mathbf X}^{T}, {\mathbf D}_{{\mathbf X},J}^{-1}, {\mathbf D}_{{\mathbf Y},I}^{-1},{\mathbf Z}_{\mathbf Y}^{T}, {\mathbf D}_{{\mathbf Y},K}^{-1} \mathrm{)}\).
The only differences between this reformulation and what we originally
established is that the generalized singular vectors (\({\mathbf P}\)
and \({\mathbf Q}\)) and the component scores
(\({\mathbf F}_{\mathbf J}\) and \({\mathbf F}_{\mathbf K}\)) differ by
constant scaling factors (which are the sums of \({\mathbf X}\) and
\({\mathbf Y}\) for their respective scores).

We can regularize PLS-CA-R in the same way as Takane's RMCA. We require
(1) a ridge parameter which we refer to as \(\epsilon\) and (2) variants
of \({\mathbf D}_{{\mathbf X},I}\), \({\mathbf D}_{{\mathbf X},J}\),
\({\mathbf D}_{{\mathbf Y},I}\), and \({\mathbf D}_{{\mathbf Y},K}\)
that we refer to as
\({\mathbb D}_{{\mathbf X},I} = {\mathbf D}_{{\mathbf X},I} + [\epsilon \times ({\mathbf Z}_{\mathbf X}{\mathbf Z}_{\mathbf X}^{T})^{+}]\),
\({\mathbb D}_{{\mathbf Y},I} = {\mathbf D}_{{\mathbf Y},I} + [\epsilon \times ({\mathbf Z}_{\mathbf Y}{\mathbf Z}_{\mathbf Y}^{T})^{+}]\),
\({\mathbb D}_{{\mathbf X},J} = {\mathbf D}_{{\mathbf X},J} + [\epsilon \times {\mathbf Z}_{\mathbf X}^{T}({\mathbf Z}_{\mathbf X}{\mathbf Z}_{\mathbf X}^{T})^{+}{\mathbf Z}_{\mathbf X}]\),
and
\({\mathbb D}_{{\mathbf Y},K} = {\mathbf D}_{{\mathbf Y},K} + [\epsilon \times {\mathbf Z}_{\mathbf Y}^{T}({\mathbf Z}_{\mathbf Y}{\mathbf Z}_{\mathbf Y}^{T})^{+}{\mathbf Z}_{\mathbf Y}]\).
When \(\epsilon = 0\) then
\({\mathbb D}_{{\mathbf X},I} = {\mathbf D}_{{\mathbf X},I}\),
\({\mathbb D}_{{\mathbf Y},I} = {\mathbf D}_{{\mathbf Y},I}\),
\({\mathbb D}_{{\mathbf X},J} = {\mathbf D}_{{\mathbf X},J}\),
\({\mathbb D}_{{\mathbf Y},K} = {\mathbf D}_{{\mathbf Y},K}\). We obtain
ridge-like regularized forms of PLS-CA for the correlation, regression,
and canonical decompositions if we replace the GPLSSVD step in (or just
the input to) each algorithm with
\(\mathrm{GPLSSVD(}{\mathbb D}_{{\mathbf X},I}^{-1},{\mathbf Z}_{\mathbf X}^{T}, {\mathbb D}_{{\mathbf X},J}^{-1}, {\mathbb D}_{{\mathbf Y},I}^{-1},{\mathbf Z}_{\mathbf Y}^{T}, {\mathbb D}_{{\mathbf Y},K}^{-1} \mathrm{)}\).
As per Takane's recommendation \citep{takane_regularized_2006},
\(\epsilon\) could be any positive value, though integers in the range
from 1 to 20 provide sufficient regularization.

However, the above approach may not be feasible when \(I\), \(J\),
and/or \(K\) are particularly large because the various crossproduct and
projection matrices require a large amount of memory and/or
computational expense. So,we can use a ``truncated'' version of the
Takane regularization which is more computationally efficient, and
analogous to the regularization procedure of Allen
\citep{allen_sparse_2013, allen_generalized_2014}. We re-define
\({\mathbb D}_{{\mathbf X},I} = {\mathbf D}_{{\mathbf X},I} + (\epsilon \times {\mathbf I})\)
and
\({\mathbb D}_{{\mathbf Y},I} = {\mathbf D}_{{\mathbf Y},I} + (\epsilon \times {\mathbf I})\)
and then also
\({\mathbb D}_{{\mathbf X},J} = {\mathbf D}_{{\mathbf X},J} + (\epsilon \times {\mathbf I})\)
and
\({\mathbb D}_{{\mathbf Y},K} = {\mathbf D}_{{\mathbf Y},K} + (\epsilon \times {\mathbf I})\)
where \({\mathbf I}\) are identity matrices of appropriate size. Like in
the previous formulation, we replace the values we have in the GPLSSVD
step where
\(\mathrm{GPLSSVD(}{\mathbb D}_{{\mathbf X},I}^{-1},{\mathbf Z}_{\mathbf X}^{T}, {\mathbb D}_{{\mathbf X},J}^{-1}, {\mathbb D}_{{\mathbf Y},I}^{-1},{\mathbf Z}_{\mathbf Y}^{T}, {\mathbb D}_{{\mathbf Y},K}^{-1} \mathrm{)}\);
and in this particular case, the weight matrices are all diagonal
matrices, which allows for a lower memory footprint and less
computational burden.

We have two concluding remarks on the ridge-like regularizations we
presented. First, though the above are presented under teh PLS-CA
frameworks, the basis of these concepts extend to any of the GPLSSVD
techniques. In particular, the more simplified Takane/Allen hybrid
approach to ridge-like regularization is easier to generally apply: it
requires only some inflation factor (i.e., \(\epsilon\)) along the
diagonals of the weight matrices. However, the first (Takane's) approach
was established in a framework more akin to CCA, and thus could also be
used for any of the optimization approaches outlined in Table
\ref{table:crossdecomp}. Second, though we presented ridge-like
regularization with a single \(\epsilon\) it is entirely possible to use
different \(\epsilon\)s for each set of weights. Although it is
possible, we do not necessarily recommend this approach, as it requires
a (potentially expensive) grid search over all the various \(\epsilon\)
parameters. Alternatively, if multiple \(\epsilon\)s were used, one
could minimize the number of parameters to search and set some of the
\(\epsilon\)s to 0 and, for example, use only one or two \(\epsilon\)
values instead of four possible \(\epsilon\) values.

\hypertarget{implementation-of-algorithms}{%
\subsection{Implementation of
algorithms}\label{implementation-of-algorithms}}

We provide an \texttt{R} package that implements all of the algorithms,
with the cross-decomposition variations here:
\url{https://github.com/derekbeaton/gpls}. This package provides direct
interfaces to PLS-COR, PLS-REG, PLS-CAN, CCA, RRR, and PLS-CA-COR,
PLS-CA-REG, and PLS-CA-CAN, as well as the generalized PLS approaches as
outlined above.

\hypertarget{further-extensions-of-pls-ca-r-and-gplssvd}{%
\section{Further extensions of PLS-CA-R and
GPLSSVD}\label{further-extensions-of-pls-ca-r-and-gplssvd}}

In both the main text and here, we have foregone any discussions of
inference, stability, and resampling for PLS-CA-R (or GPLSSVD) in part
because many of the inference and stability approaches established
throughout the broader PLS (and CCA) literature still apply with little
or no changes. Such approaches include feature selection or
sparsification \citep{sutton_sparse_2018}, additional regularization or
sparsification approaches
\citep{le_floch_significant_2012-1, guillemot2019constrained, tenenhaus_variable_2014, tenenhaus_regularized_2011},
cross-validation
\citep{wold_principal_1987, rodriguez-perez_overoptimism_2018, kvalheim_number_2019, abdi_partial_2010-1},
permutation \citep{berry_permutation_2011, winkler2020permutation},
various bootstrap approaches
\citep{abdi_partial_2010-1, takane_regularized_2009-1} or tests
\citep{mcintosh_partial_2004, krishnan_partial_2011}, and other
frameworks such as split-half resampling
\citep{strother_quantitative_2002-1, kovacevic2013revisiting, strother2004optimizing}.

\bibliographystyle{agsm}
\bibliography{plscar.bib}

\end{document}
