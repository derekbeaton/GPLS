% !TeX program = pdfLaTeX
\documentclass[12pt]{article}
\usepackage{amsmath}
\usepackage{graphicx,psfrag,epsf}
\usepackage{enumerate}
\usepackage{natbib}
\usepackage{textcomp}
\usepackage[hyphens]{url} % not crucial - just used below for the URL
\usepackage{hyperref}
\providecommand{\tightlist}{%
  \setlength{\itemsep}{0pt}\setlength{\parskip}{0pt}}

%\pdfminorversion=4
% NOTE: To produce blinded version, replace "0" with "1" below.
\newcommand{\blind}{0}

% DON'T change margins - should be 1 inch all around.
\addtolength{\oddsidemargin}{-.5in}%
\addtolength{\evensidemargin}{-.5in}%
\addtolength{\textwidth}{1in}%
\addtolength{\textheight}{1.3in}%
\addtolength{\topmargin}{-.8in}%

%% load any required packages here




\usepackage{float}
\usepackage{bbold}
\usepackage{subfig}
\usepackage{graphicx}
\usepackage[utf8]{inputenc}
\usepackage[T1]{fontenc}
\usepackage{booktabs}
\usepackage{algorithm2e}
\usepackage{caption}
\usepackage{tabularx}
\usepackage{verbatim}
\usepackage{xcolor}

\begin{document}


\def\spacingset#1{\renewcommand{\baselinestretch}%
{#1}\small\normalsize} \spacingset{1}


%%%%%%%%%%%%%%%%%%%%%%%%%%%%%%%%%%%%%%%%%%%%%%%%%%%%%%%%%%%%%%%%%%%%%%%%%%%%%%

\if0\blind
{
  \title{\bf A generalization of partial least squares regression and correspondence
analysis for categorical and mixed data: An application with the ADNI
data}

  \author{
        Derek Beaton \\
    Rotman Research Institute, Baycrest Health Sciences\\
     and \\     ADNI \thanks{Data used in preparation of this article were obtained from the
Alzheimer's Disease Neuroimaging Initiative (ADNI) database
(\url{http://adni.loni.usc.edu/}). As such, the investigators within the
ADNI contributed to the design and implementation of ADNI and/or
provided data but did not participate in analysis or writing of this
report. A complete listing of ADNI investigators can be found at
http://adni.loni.ucla.edu/wpcontent/uploads/how\_to\_apply/ADNI\_Acknowledgement\_List.pdf} \\
    ADNI\\
     and \\     Gilbert Saporta \\
    Conservatoire National des Arts et Metiers\\
     and \\     Hervé Abdi \\
    Behavioral and Brain Sciences, The University of Texas at Dallas\\
      }
  \maketitle
} \fi

\if1\blind
{
  \bigskip
  \bigskip
  \bigskip
  \begin{center}
    {\LARGE\bf A generalization of partial least squares regression and correspondence
analysis for categorical and mixed data: An application with the ADNI
data}
  \end{center}
  \medskip
} \fi

\bigskip
\begin{abstract}
Current large scale studies of brain and behavior typically involve
multiple populations, diverse types of data (e.g., genetics, brain
structure, behavior, demographics, or ``mutli-omics,'' and
``deep-phenotyping'') measured on various scales of measurement. To
analyze these heterogeneous data sets we need simple but flexible
methods able to integrate the inherent properties of these complex data
sets. Here we introduce partial least squares-correspondence
analysis-regression (PLS-CA-R) a method designed to address these
constraints. PLS-CA-R generalizes PLS regression to most data types
(e.g., continuous, ordinal, categorical, non-negative values). We also
show that PLS-CA-R generalizes many ``two-table'' multivariate
techniques and their respective algorithms, such as various PLS
approaches, canonical correlation analysis, and redundancy analysis
(a.k.a. reduced rank regression).
\end{abstract}

\noindent%
{\it Keywords:} generalized singular value decomposition, latent models, genetics, neuroimaging, canonical correlation analysis
\vfill

\newpage
\spacingset{1.45} % DON'T change the spacing!

\hypertarget{introduction}{%
\section{Introduction}\label{introduction}}

\label{section:Intro}

Today's large scale and multi-site studies, such as the UK BioBank
(\url{https://www.ukbiobank.ac.uk/}) and the Rotterdam study
(\url{http://www.erasmus-epidemiology.nl/}), collect population level
data across numerous types and modalities, (e.g., genetics,
neurological, behavioral, clinical, and laboratory measures). Other
types of large scale studies---typically those that emphasize diseases
and disorders---involve a relatively small of participants but collect a
very large number of measures on diverse modalities. Some such studies
include the Ontario Neurodegenerative Disease Research Initiative
(ONDRI) \citep{farhan_ontario_2016} which includes genetics, multiple
types of magnetic resonance brain imaging
\citep{duchesne_canadian_2019}, a wide array of behavioral, cognitive,
clinical, and laboratory batteries, as well as many modalities
``between'' these types, such as ocular imaging, gait, and balance
\citep{montero-odasso_motor_2017-1}, eye tracking, and neuro-pathology.
Though large samples (e.g., UK BioBank) and depth of data (e.g., ONDRI)
are necessary to understand typical and disordered samples and
populations, few statistical or machine learning approaches exist 1)
that can accommodate such large (whether ``big'' or ``wide''), complex,
and heterogeneous data sets and 2) that also respect the inherent
properties of such data.

In many cases, the analysis of a mixture of data types loses information
in part because the data need to be transformed to fit the analytical
methods; but this analysis also loses inferential power because the
standard assumptions may be inappropriate or incorrect. For example, to
analyze categorical and continuous data together, a typical---but
inappropriate---strategy is to recode the continuous data into
categories such as dichotomization, trichotomization, or other (often
arbitrary) binning strategies. Furthermore, ordinal and Likert scale
data---such as responses on many cognitive, behavioral, clinical, and
survey instruments---are often incorrectly treated as metric or
continuous values \citep{burkner_ordinal_nodate}. When it comes to
genetic data, such as single nucleotide polymorphims (SNPs), the data
are almost always recoded by counting the number of the minor homozygote
to give: 0 for the major homozygote (two copies of the most frequent of
the two alleles), 1 for the heterozygote, and 2 for the minor
homozygote. This \{0, 1, 2\} recoding of genotypes (1) assumes additive
linear effects based on the minor homozygote and (2) is often treated as
metric/continuous values (as opposed to categorical or ordinal), even
when known effects of risk are neither linear nor additive, such as
haplotypic effects \citep{vormfelde_value_2007} nor exclusively based on
the minor homozygotes, such as ApoE in Alzheimer's Disease
\citep{genin_apoe_2011}. Interestingly other, less restrictive, models
(e.g., dominant, recessive, genotypic) perform better
\citep{lettre2007genetic} than the additive model, but these are rarely
used.

Here we introduce partial least squares-correspondence
analysis-regression (PLS-CA-R): a latent variable regression modeling
approach suited for the analysis of complex data sets. We first show
that PLS-CA-R generalizes PLS regression
\citep{wold_soft_1975, wold_collinearity_1984, tenenhaus_regression_1998, abdi_partial_2010-1},
CA
\citep{greenacre_theory_1984, greenacre_correspondence_2010-1, lebart_multivariate_1984},
and PLS-CA \citep{beaton_partial_2016}---which is the ``correlation''
companion and basis of PLS-CA-R. We then show that PLS-CA-R is a
data-type general PLS regression method based on the generalized
singular value decomposition (GSVD) that combines the features of CA (to
accommodate multiple data types) with the features of PLS-R (as a
regression method generalizing ordinary least squares regression when
its assumptions are not met).

We illustrate the multiple uses of PLS-CA-R, with data from the
Alzheimer's Disease Neuroimaging Initiative (ADNI). These data include:
diagnosis data (mutually exclusive categories), SNPs (genotypes are
categorical), multiple behavioral or clinical instruments (that could be
ordinal, categorical, or continuous), and several neuroimaging measures
and indices (generally either continuous or non-negative). PLS-CA-R can
1) accommodate different data types in a predictive or fitting
framework, 2) regress out (i.e., residualize) effects of mixed and
collinear data, and 3) reveal latent variables within a well-established
framework.

In addition, we provide an appendix that shows how PLS-CA-R provides the
basis for further generalizations. These include different optimization
schemes (e.g., covariance as in PLS or correlation as in canonical
correlation), transformations for alternate metrics, ridge-like
regularization, and three types of PLS algorithms (regression,
correlation, and canonical). We also provide an R package and examples
at \url{https://github.com/derekbeaton/gpls}.

This paper is organized as follows. In Section \ref{section:PLSCAR} we
introduce PLS-CA-R. Next, in Section \ref{section:appex}, we illustrate
PLS-CA-R on the TADPOLE challenge
(\url{https://tadpole.grand-challenge.org/}) and additional genetics
data from ADNI across three examples: 1) a simple discriminant example
with categorical data, 2) a mixed data example that requires
residualization, and 3) a larger example of multiple genetic markers and
whole brain tracer uptake (non-negative values). Finally in Section
\ref{section:Disc} we provide a discussion and conclusions.

\hypertarget{partial-least-squares-correspondence-analysis-regression}{%
\section{Partial least squares-correspondence
analysis-regression}\label{partial-least-squares-correspondence-analysis-regression}}

\label{section:PLSCAR}

Here we present the generalization of partial least square-regression
(PLS-R) to multiple correspondence analysis (MCA) and correspondence
analysis (CA)-like methods that generally apply to categorical (nominal)
data. Via CA, we can also generalize to other data types including mixed
types (e.g., categorical, ordinal, continuous, contingency).

PLS ``regression''
\citep{wold_soft_1975, wold_collinearity_1984, wold_pls-regression_2001}.
PLS regression decomposition is an asymmetric and iterative method. The
asymmetry is that one data table is privileged (i.e., treated as
predictors). PLSR computes its solution iteratively---via alternating
least squares, the SVD, or similar techniques---and makes use of
deflation for both data tables.

\hypertarget{notation}{%
\subsection{Notation}\label{notation}}

Bold uppercase letters denote matrices (e.g., \(\mathbf{X}\)), bold
lowercase letters denote vectors (e.g., \(\bf{x}\)), and italic
lowercase letters denote specific elements (e.g., \(x\)). Upper case
italic letters (e.g., \(I\)) denote cardinality, size, or length where a
lower case italic (e.g., \(i\)) denotes a specific value of an index. A
generic element of matrix \(\mathbf{X}\) would be denoted \(x_{i,j}\).
Common letters of varying type faces, for example \({\bf X}\),
\(\bf{x}\), \(x_{i,j}\), come from the same data structure. A
preprocessed or transformed version of a matrix \({\mathbf X}\) is
denoted \({\mathbf Z}_{\mathbf X}\). Vectors are assumed to be column
vectors unless otherwise specified. Two matrices side-by-side denotes
standard matrix multiplication (e.g., \(\bf{X}\bf{Y}\)). The operator
\(\odot\) denotes element-wise (Hadamard) multiplication and \(\oslash\)
denotes element-wise (Hadamard) division. The matrix \({\bf I}\) denotes
the identity matrix, \(\mathbf{1}\) denotes a vector 1's and
\({\mathbf 0}\) is a null matrix (all entries are \(0\)). Superscript
\(^{T}\) denotes the transpose operation, superscript \(^{-1}\) denotes
standard matrix inversion, and superscript \(^{+}\) denotes the
Moore-Penrose pseudo-inverse. The diagonal operator, denoted
\(\mathrm{diag\{\}}\), transforms a vector into a diagonal matrix, or
extracts the diagonal of a matrix to produce a vector.

\hypertarget{the-generalized-svd-and-correspondence-analysis}{%
\subsection{The Generalized SVD and Correspondence
Analysis}\label{the-generalized-svd-and-correspondence-analysis}}

\label{section:GSVDCA}

Given an \(I \times J\) matrix \({\mathbf X}\), the singular value
decomposition (SVD) decomposes \({\mathbf X}\) as
\begin{equation}\label{eq:svd}
{\mathbf Z}_{\mathbf X} = 
{\mathbf U} {\boldsymbol \Delta} {\mathbf V}^{T}
\textrm{ with } {\mathbf U}^{T}{\mathbf U} 
= {\mathbf I} = {\mathbf V}^{T}{\mathbf V},
\end{equation} where \({\mathbf X}\) is of rank \(A\) (with
\(A \leq \min(I,J)\)) and \({\boldsymbol \Delta}\) is the \(A \times A\)
diagonal matrix of the singular values, and
\({\boldsymbol \Lambda} = {\boldsymbol \Delta}^2\) is the \(A \times A\)
diagonal matrix of the eigenvalues.

The \emph{generalized} singular value decomposition (GSVD) generalizes
the SVD by integrating constraints---also called \emph{metrics} because
these matrices define a metric in a generalized Euclidean space---
imposed on the rows and columns of the matrix to be decomposed.
Specifically, when both an \(I\times I\) matrix \({\mathbf M}\) and a
\(J\times J\) matrix \({\mathbf W}\) positive definite matrices, the
GSVD decomposes \({\mathbf X}\) as \begin{equation}\label{eq:gsvd}
{\mathbf X} = {\mathbf P}{\boldsymbol \Delta} {\mathbf Q}^{T} 
\textrm{ with }
{\mathbf P}^{T}{\mathbf M}{\mathbf P} = {\mathbf I} = {\mathbf Q}^{T}{\mathbf W}{\mathbf Q}
\end{equation} where \({\mathbf X}\) is of rank \(A\) and
\({\boldsymbol \Delta}\) is the \(A \times A\) diagonal matrix of the
singular values, \({\boldsymbol \Lambda} = {\boldsymbol \Delta}^2\) is
the \(A \times A\) diagonal matrix of the eigenvalues, and
\({\mathbf P}\) and \({\mathbf Q}\) are the \emph{generalized} singular
vectors. We compute the typical singular vectors as
\({\mathbf U} = {\mathbf M}^{\frac{1}{2}}{\mathbf P}\) and
\({\mathbf V} = {\mathbf W}^{\frac{1}{2}}{\mathbf Q}\), where
\({\mathbf U}^{T}{\mathbf U} = {\mathbf I} = {\mathbf V}^{T}{\mathbf V}\).
From the weights, generalized singular vectors, and singular values we
can obtain component (a.k.a. factor) scores as
\({\mathbf F}_{I} = {\mathbf M}{\mathbf P}{\boldsymbol \Delta}\) and
\({\mathbf F}_{J} = {\mathbf W}{\mathbf Q}{\boldsymbol \Delta}\) for the
\(I\) rows and \(J\) columns of \({\mathbf X}\), respectively. For
convenience, the GSVD is presented as a ``triplet'' with the row metric
(weights), data, and column metric (weights) as
\(\mathrm{GSVD(}{\mathbf M}, {\mathbf X}, {\mathbf W}\mathrm{)}\). We
present the GSVD triplet here in slightly different way than it is
typically presented \citep[see, e.g.,][]{holmes_multivariate_2008}. Our
presentation of the GSVD reflects the multiplication steps taken before
decomposition (see our Appendix).

Correspondence analysis (CA) is a technique akin to principal components
analysis (PCA)---originally designed for the analysis of contingency
tables---and analyzes deviations to independence. See
\citet{greenacre_theory_1984}, \citet{greenacre_correspondence_2010-1},
and \citet{lebart_multivariate_1984} for detailed explanations of CA,
and then see \citet{escofier-cordier_analyse_1965} and
\citet{benzecri_analyse_1973} for the origins and early developments of
CA. CA analyzes an \(I \times J\) matrix \({\mathbf X}\) whose entries
are all non-negative. CA is performed with the GSVD as follows. First
the \emph{observed} probabilities matrix is computed as
\({\mathbf O}_{\mathbf X} = {\mathbf X} \times ({\mathbf 1}^{T}{\mathbf X} {\mathbf 1})^{-1}\)
The marginal probabilities from the observed matrix are then computed as
\begin{equation}
{\mathbf m}_{\mathbf X} = {\mathbf O}_{\mathbf X}{\mathbf 1} 
\text{ and } 
{\mathbf w}_{\mathbf X} = ({\mathbf 1}^{T}{\mathbf O}_{\mathbf X})^{T}.
\end{equation} The \emph{expected} probabilities matrix is computed as
\({\mathbf E}_{\mathbf X} = {\mathbf m}_{\mathbf X}{\mathbf w}_{\mathbf X}^{T}\).
The \emph{deviations} from independence matrix is then \begin{equation}
{\mathbf Z}_{\mathbf X} 
= {\mathbf O}_{\mathbf X} - {\mathbf E}_{\mathbf X}.
\end{equation} We compute CA from
\(\mathrm{GSVD(} {\mathbf M}_{\mathbf X}^{-1}, {\mathbf Z}_{\mathbf X}, {\mathbf W}_{\mathbf X}^{-1} \mathrm{)}\)
with
\({\mathbf M}_{\mathbf X} = \mathrm{diag\{} {\mathbf m}_{\mathbf X} \mathrm{\}}\)
and
\({\mathbf W}_{\mathbf X} = \mathrm{diag\{} {\mathbf w}_{\mathbf X} \mathrm{\}}\).

\begin{table}[!h]

\caption{\label{tab:unnamed-chunk-1}\label{table:disj} An example of disjunctive (SEX) and pseudo-disjunctive (AGE, EDU) coding through the Escofier or fuzzy transforms. For disjunctive an pseudo-disunctive data, each variable has a row-wise sum of 1 across its respective columns, and thus the row sums across the table are the number of original variables.}
\centering
\begin{tabular}[t]{llrrrrrrrr}
\toprule
\multicolumn{1}{c}{ } & \multicolumn{3}{c}{Original coding} & \multicolumn{6}{c}{Disjunctive and pseudo-disjunctive coding} \\
\cmidrule(l{2pt}r{2pt}){2-4} \cmidrule(l{2pt}r{2pt}){5-10}
\multicolumn{4}{c}{ } & \multicolumn{2}{c}{SEX} & \multicolumn{2}{c}{AGE} & \multicolumn{2}{c}{EDU} \\
\cmidrule(l{2pt}r{2pt}){5-6} \cmidrule(l{2pt}r{2pt}){7-8} \cmidrule(l{2pt}r{2pt}){9-10}
  & SEX & AGE & EDU & Male & Female & AGE- & AGE+ & EDU- & EDU+\\
\midrule
SUBJ 1 & Male & 64.8 & 16 & 1 & 0 & 1.03 & -0.03 & 0.33 & 0.67\\
SUBJ 2 & Female & 63.6 & 18 & 0 & 1 & 1.11 & -0.11 & 0.17 & 0.83\\
SUBJ 3 & Female & 76.4 & 18 & 0 & 1 & 0.24 & 0.76 & 0.17 & 0.83\\
SUBJ 4 & Male & 66.0 & 18 & 1 & 0 & 0.95 & 0.05 & 0.17 & 0.83\\
SUBJ 5 & Female & 61.9 & 14 & 0 & 1 & 1.23 & -0.23 & 0.50 & 0.50\\
\addlinespace
SUBJ 6 & Female & 66.7 & 14 & 0 & 1 & 0.90 & 0.10 & 0.50 & 0.50\\
\bottomrule
\end{tabular}
\end{table}

\hypertarget{pls-ca-r}{%
\subsection{PLS-CA-R}\label{pls-ca-r}}

\label{section:plscar_form}

For simplicity assume in the following formulation that \({\mathbf X}\)
and \({\mathbf Y}\) are both complete disjunctive tables as seen in
Table \ref{table:disj} (see SEX columns) or Table
\ref{table:snps_models_disj}. This formulation also applies generally to
non-negative data (see later sections). We define observed matrices for
\({\mathbf X}\) and \({\mathbf Y}\) as

\begin{equation}
\begin{aligned}
{\mathbf O}_{\mathbf X} = {\mathbf X} \times ({\mathbf 1}^{T}{\mathbf X} {\mathbf 1})^{-1}, \\
{\mathbf O}_{\mathbf Y} = {\mathbf Y} \times ({\mathbf 1}^{T}{\mathbf Y} {\mathbf 1})^{-1}
\label{eq:observedXY}
\end{aligned}
\end{equation} Next we compute marginal probabilities for the rows and
columns. We compute row marginal probabilities as \begin{equation}
{\mathbf m}_{\mathbf X} = {\mathbf O}_{\mathbf X}{\mathbf 1} \text{ and } {\mathbf m}_{\mathbf Y} = {\mathbf O}_{\mathbf Y}{\mathbf 1},
\label{eq:xy_rowvecs}
\end{equation} and column probabilities as \begin{equation}
{\mathbf w}_{\mathbf X} = ({\mathbf 1}^{T}{\mathbf O}_{\mathbf X})^{T} \text{ and } {\mathbf w}_{\mathbf Y} = ({\mathbf 1}^{T}{\mathbf O}_{\mathbf Y})^{T},
\label{eq:weightmats_v1}
\end{equation} We then define expected matrices as \begin{equation}
{\mathbf E}_{\mathbf X} = {\mathbf m}_{\mathbf X}{\mathbf w}_{\mathbf X}^{T} \text{ and } {\mathbf E}_{\mathbf Y} = {\mathbf m}_{\mathbf Y}{\mathbf w}_{\mathbf Y}^{T},
\label{eq:models}
\end{equation} and deviation matrices as \begin{equation}
{\mathbf Z}_{\mathbf X} = {\mathbf O}_{\mathbf X} - {\mathbf E}_{\mathbf X} \text{ and } {\mathbf Z}_{\mathbf Y} = {\mathbf O}_{\mathbf Y} - {\mathbf E}_{\mathbf Y},
\label{eq:plscar_Zs}
\end{equation}

For PLS-CA-R we have row and column weights of
\({\mathbf M}_{\mathbf X} = \mathrm{diag\{}{\mathbf m}_{\mathbf X} \mathrm{\}}\),
\({\mathbf M}_{\mathbf Y} = \mathrm{diag\{}{\mathbf m}_{\mathbf Y} \mathrm{\}}\),
\({\mathbf W}_{\mathbf X} = \mathrm{diag\{}{\mathbf w}_{\mathbf X} \mathrm{\}}\),
and
\({\mathbf W}_{\mathbf Y} = \mathrm{diag\{}{\mathbf w}_{\mathbf Y} \mathrm{\}}\).
PLS-CA-R makes use of the rank 1 SVD solution iteratively and works as
cross-product
\({\mathbf Z}_{\mathbf R} = ({\mathbf M}_{\mathbf X}^{-\frac{1}{2}}{\mathbf Z}_{\mathbf X}{\mathbf W}_{\mathbf X}^{-\frac{1}{2}})^{T}({\mathbf M}_{\mathbf Y}^{-\frac{1}{2}}{\mathbf Z}_{\mathbf Y}{\mathbf W}_{\mathbf Y}^{-\frac{1}{2}})\),
where \begin{equation}
{\mathbf Z}_{\mathbf R} = {\mathbf U} {\boldsymbol \Delta} {\mathbf V}^{T},
\label{eq:the_svd_for_plscar}
\end{equation} where
\({\bf U}^{T}{\bf U} = {\bf I} = {\bf V}^{T}{\bf V}\). We can obtain the
\emph{generalized} singular vectors as
\({\mathbf P} = {\mathbf W}_{{\mathbf X}}^{\frac{1}{2}}{\mathbf U}\) and
\({\mathbf Q} = {\mathbf W}_{{\mathbf X}}^{\frac{1}{2}}{\mathbf V}\),
where
\({\bf P}^{T}{\bf W}_{\bf X}^{-1}{\bf P} = {\bf I} = {\bf Q}^{T}{\bf W}_{\bf Y}^{-1}{\bf Q}\).
Because we make use of the rank 1 solution iteratively, we only retain
the first vectors and values from Eq. \ref{eq:the_svd_for_plscar}. We
distinguish the retained vectors and values as \(\tilde\delta\),
\(\widetilde{\mathbf u}\), \(\widetilde{\mathbf v}\),
\(\widetilde{\mathbf p}\), and \(\widetilde{\mathbf q}\). At each
iteration, we compute the component scores, for the \(J\) columns of
\({\bf Z}_{\bf X}\) and the \(K\) columns of \({\bf Z}_{\bf Y}\),
respectively, as
\(\widetilde{\mathbf f}_{J} = {\bf W}_{\bf X}^{-1}\widetilde{\mathbf p}{\widetilde\delta}\)
and
\(\widetilde{\mathbf f}_{K} = {\bf W}_{\bf Y}^{-1}\widetilde{\mathbf q}{\widetilde\delta}\).
We compute the latent variables as \begin{equation}
\begin{aligned}
{\boldsymbol \ell}_{\mathbf X} = ({\mathbf M}_{\mathbf X}^{-\frac{1}{2}}{\mathbf Z}_{\mathbf X}{\mathbf W}_{\mathbf X}^{-\frac{1}{2}})\widetilde{\mathbf u} \text{ and } {\boldsymbol \ell}_{\mathbf Y} = ({\mathbf M}_{\mathbf Y}^{-\frac{1}{2}}{\mathbf Z}_{\mathbf Y}{\mathbf W}_{\mathbf Y}^{-\frac{1}{2}})\widetilde{\mathbf v}.
\label{eq:lvs}
\end{aligned}
\end{equation} Next we compute
\({\mathbf t}_{\mathbf X} = {\boldsymbol \ell}_{\mathbf X} \times {{\lvert\lvert {\boldsymbol \ell}_{\mathbf X} \rvert\rvert}^{-1}}\),
\(b = {\boldsymbol \ell}_{\mathbf Y}^{T}{\mathbf t}_{\mathbf X}\), and
\(\widehat{\mathbf u} = {\mathbf t}_{\mathbf X}^{T} ({\mathbf M}_{\mathbf X}^{-\frac{1}{2}}{\mathbf Z}_{\mathbf X}{\mathbf W}_{\mathbf X}^{-\frac{1}{2}})\).
We use \({\mathbf t}_{\mathbf X}\), \(b\), and \(\widehat{\mathbf u}\)
to compute rank 1 reconstructed versions of \({\mathbf Z}_{\mathbf X}\)
and \({\mathbf Z}_{\mathbf Y}\) as

\begin{equation}
\widehat{\mathbf Z}_{{\mathbf X},1} = {\mathbf M}_{\mathbf X}^{\frac{1}{2}}({\mathbf t}_{\mathbf X}\widehat{\mathbf u}^{T}){\mathbf W}_{\mathbf X}^{\frac{1}{2}} \text{ and } \widehat{\mathbf Z}_{{\mathbf Y},1} = {\mathbf M}_{\mathbf Y}^{\frac{1}{2}}(b{\mathbf t}_{\mathbf X}\widetilde{\mathbf v}^{T}){\mathbf W}_{\mathbf Y}^{\frac{1}{2}}.
\label{eq:rank1_preds_plscar}
\end{equation} Finally, we deflate \({\mathbf Z}_{\mathbf X}\) and
\({\mathbf Z}_{\mathbf Y}\) as
\({\mathbf Z}_{\mathbf X} = {\mathbf Z}_{\mathbf X} - \widehat{\mathbf Z}_{{\mathbf X},1}\)
and
\({\mathbf Z}_{\mathbf Y} = {\mathbf Z}_{\mathbf Y} - \widehat{\mathbf Z}_{{\mathbf Y},1}\).
We then repeat the iterative procedure with these deflated
\({\mathbf Z}_{\mathbf X}\) and \({\mathbf Z}_{\mathbf Y}\). The
computations outlined above are performed for \(C\) iterations where:
(1) \(C\) is some pre-specified number of intended latent variables
where \(C \leq A\) where \(A\) is the rank of
\({\bf M}_{\bf X}^{-\frac{1}{2}}{\mathbf Z}_{\mathbf X}{\bf W}_{\bf X}^{-\frac{1}{2}}\)
or (2) when \({\mathbf Z}_{\mathbf X} = {\mathbf 0}\) or
\({\mathbf Z}_{\mathbf Y} = {\mathbf 0}\). Upon the stopping condition
we would have \(C\) components, and would have collected any vectors
into corresponding matrices. Those matrices are

\begin{itemize}
\item
  two \(C \times C\) diagonal matrices \({\mathbf B}\) and
  \(\widetilde{\boldsymbol \Delta}\) with each \(b\) and
  \(\tilde\delta\) on the diagonal with zeros off-diagonal,
\item
  the \(I \times C\) matrices \({\mathbf L}_{\mathbf X}\),
  \({\mathbf L}_{\mathbf Y}\), and \({\mathbf T}_{\mathbf X}\),
\item
  the \(J \times C\) matrices \(\widetilde{\mathbf U}\),
  \(\widehat{\mathbf U}\), \(\widetilde{\mathbf P}\), and
  \(\widetilde{\mathbf F}_{J}\), and
\item
  the \(K \times C\) matrices \(\widetilde{\mathbf V}\),
  \(\widetilde{\mathbf Q}\), \(\widetilde{\mathbf F}_{K}\).
\end{itemize}

Algorithm \ref{algo:plscar} shows the key elements of the PLS-CA-R
algorithm, from input to deflation.

\RestyleAlgo{boxed}
\begin{algorithm}
\DontPrintSemicolon
\SetAlgoLined
\KwResult{PLS-CA-R between ${\mathbf Z}_{{\mathbf X}}$ and ${\mathbf Z}_{{\mathbf Y}}$}
\SetKwInOut{Input}{Input}\SetKwInOut{Output}{Output}
\Input{${\mathbf M}_{\mathbf{X}}$, ${\mathbf Z}_{{\mathbf X}}$, ${\mathbf W}_{\mathbf{X}}$, ${\mathbf M}_{\mathbf{Y}}$, ${\mathbf Z}_{{\mathbf Y}}$, ${\mathbf W}_{\mathbf{Y}}$, $C$}
\Output{$\widetilde{\boldsymbol \Delta}$, $\widetilde{\mathbf U}$, $\widetilde{\mathbf V}$, $\widetilde{\mathbf P}$, $\widetilde{\mathbf Q}$, $\widetilde{\mathbf F}_{J}$, $\widetilde{\mathbf F}_{K}$, ${\mathbf L}_{{\mathbf X}}$, ${\mathbf L}_{{\mathbf Y}}$, ${\mathbf T}_{{\mathbf X}}$, $\widehat{\mathbf U}$, ${\mathbf B}$}
\BlankLine
\For{$c=1, \dots, C$}{

  ${\mathbf Z}_{\mathbf R} \leftarrow ({\mathbf M}_{\mathbf X}^{-\frac{1}{2}}{\mathbf Z}_{\mathbf X}{\mathbf W}_{\mathbf X}^{-\frac{1}{2}})^{T}({\mathbf M}_{\mathbf Y}^{-\frac{1}{2}}{\mathbf Z}_{\mathbf Y}{\mathbf W}_{\mathbf Y}^{-\frac{1}{2}})$ \\
  Compute rank 1 SVD of ${\mathbf Z}_{\mathbf R}$, retain $\widetilde{\bf u}, \widetilde{\bf v}$ and $\widetilde{\delta}$ \\
  $\widetilde{\bf p} \leftarrow {\mathbf W}_{{\mathbf X}}^{\frac{1}{2}}\widetilde{\mathbf u}$ \\
  $\widetilde{\bf q} \leftarrow {\mathbf W}_{{\mathbf Y}}^{\frac{1}{2}}\widetilde{\mathbf v}$ \\
  $\widetilde{\mathbf f}_{J} \leftarrow {\bf W}_{\bf X}^{-1}\widetilde{\mathbf p}{\widetilde\delta}$ \\
  $\widetilde{\mathbf f}_{K} \leftarrow {\bf W}_{\bf Y}^{-1}\widetilde{\mathbf q}{\widetilde\delta}$ \\
  ${\boldsymbol \ell}_{\mathbf X} \leftarrow ({\mathbf M}_{\mathbf X}^{-\frac{1}{2}}{\mathbf Z}_{\mathbf X}{\mathbf W}_{\mathbf X}^{-\frac{1}{2}})\widetilde{\mathbf u}$ \\
  ${\boldsymbol \ell}_{\mathbf Y} \leftarrow ({\mathbf M}_{\mathbf Y}^{-\frac{1}{2}}{\mathbf Z}_{\mathbf Y}{\mathbf W}_{\mathbf Y}^{-\frac{1}{2}})\widetilde{\mathbf v}$ \\
  ${\mathbf t}_{\mathbf X} \leftarrow {\boldsymbol \ell}_{\mathbf X} \times {{\lvert\lvert {\boldsymbol \ell}_{\mathbf X} \rvert\rvert}^{-1}}$\\
  $b \leftarrow {\boldsymbol \ell}_{\mathbf Y}^{T}{\mathbf t}_{\mathbf X}$ \\
  $\widehat{\mathbf u} \leftarrow {\mathbf t}_{\mathbf X}^{T} ({\mathbf M}_{\mathbf X}^{-\frac{1}{2}}{\mathbf Z}_{\mathbf X}{\mathbf W}_{\mathbf X}^{-\frac{1}{2}})$\\
  ${\mathbf Z}_{{\mathbf X}} \leftarrow {\mathbf Z}_{{\mathbf X}} - [{\mathbf M}_{\mathbf X}^{\frac{1}{2}}({\mathbf t}_{\mathbf X}\widehat{\mathbf u}^{T}){\mathbf W}_{\mathbf X}^{\frac{1}{2}}]$\\
  ${\mathbf Z}_{{\mathbf Y}} \leftarrow {\mathbf Z}_{{\mathbf Y}} - [{\mathbf M}_{\mathbf Y}^{\frac{1}{2}}(b{\mathbf t}_{\mathbf X}\widetilde{\mathbf{v}}^{T}){\mathbf W}_{\mathbf Y}^{\frac{1}{2}}]$
}
\caption{PLS-CA-R algorithm. The results of a rank 1 solution are used to compute the latent variables and values necessary for deflation of ${\mathbf Z}_{{\mathbf X}}$ and ${\mathbf Z}_{{\mathbf Y}}$.}
\label{algo:plscar}
\end{algorithm}

\hypertarget{maximization-in-pls-ca-r}{%
\subsection{Maximization in PLS-CA-R}\label{maximization-in-pls-ca-r}}

PLS-CA-R maximizes the common information between
\({\mathbf Z}_{\mathbf X}\) and \({\mathbf Z}_{\mathbf Y}\) such that

\begin{equation}
\begin{aligned}
\underset{{\boldsymbol \ell}_{\mathbf X},{\boldsymbol \ell}_{\mathbf Y}}{\operatorname{arg\,max}} = {\boldsymbol \ell}_{\mathbf X}^{T}{\boldsymbol \ell}_{\mathbf Y} = \\
\widetilde{\mathbf u}^{T}({\mathbf M}_{\mathbf X}^{-\frac{1}{2}}{\mathbf Z}_{\mathbf X}{\mathbf W}_{\mathbf X}^{-\frac{1}{2}})^{T}({\mathbf M}_{\mathbf Y}^{-\frac{1}{2}}{\mathbf Z}_{\mathbf Y}{\mathbf W}_{\mathbf Y}^{-\frac{1}{2}})\widetilde{\mathbf v} = \\
\widetilde{\mathbf u}{\mathbf Z}_{\mathbf R}\widetilde{\mathbf v} = \widetilde{\mathbf u}{\mathbf U}{\boldsymbol \Delta}{\mathbf V}\widetilde{\mathbf v}  = \widetilde\delta,
\end{aligned}
\end{equation} where \(\widetilde\delta\) is the first singular value
from \({\mathbf \Delta}\) for each \(c\) step. PLS-CA-R maximization is
subject to the orthogonality constraint that
\({\boldsymbol \ell}_{{\mathbf X},c}^{T}{\boldsymbol \ell}_{{\mathbf X},c'} = 0\)
when \(c \neq c'\). This orthogonality constraint propagates through to
many of the vectors and matrices associated with
\({\mathbf Z}_{\mathbf X}\) where
\({\mathbf T}_{\mathbf X}^{T}{\mathbf T}_{\mathbf X} = \widetilde{\mathbf P}^{T}{\mathbf W}_{\bf X}^{-1}\widetilde{\mathbf P} = \widetilde{\mathbf U}^{T}\widetilde{\mathbf U} = {\mathbf I}\);
these orthogonality constraints do not apply to the various vectors and
matrices associated with \({\mathbf Y}\).

\hypertarget{decomposition-and-reconstitution}{%
\subsection{Decomposition and
reconstitution}\label{decomposition-and-reconstitution}}

\label{section:recresp}

PLS-CA-R is a ``double decomposition'' where \begin{equation}
\begin{aligned}
{\mathbf Z}_{\mathbf X} = {\mathbf M}^{\frac{1}{2}}_{\mathbf X}{\mathbf T}\widehat{\mathbf U}^{T}{\mathbf W}^{\frac{1}{2}}_{\mathbf X} \text{ and }\\
\widehat{{\mathbf Z}}_{\mathbf Y} = {\mathbf M}^{\frac{1}{2}}_{\mathbf Y}{\mathbf T}_{\mathbf X}{\mathbf B}\widetilde{\mathbf V}^{T}{\mathbf W}^{\frac{1}{2}}_{\mathbf Y} = {\mathbf M}^{\frac{1}{2}}_{\mathbf Y}\widetilde{\mathbf Z}_{\mathbf X}\widehat{\mathbf U}^{{T}{+}}{\mathbf B}\widetilde{\mathbf V}^{T}{\mathbf W}^{\frac{1}{2}}_{\mathbf Y},
\label{eq:doubledecomp}
\end{aligned}
\end{equation} where
\(\widetilde{\mathbf Z}_{\mathbf X} = {\mathbf T}\widehat{\mathbf U}^{T} = {\mathbf M}_{\mathbf X}^{-\frac{1}{2}}{\mathbf Z}_{\mathbf X}{\mathbf W}_{\mathbf X}^{-\frac{1}{2}}\).
PLS-CA-R, like PLS-R, provides the same values as OLS where
\begin{equation}
\widehat{{\mathbf Z}}_{\mathbf Y} = {\mathbf M}^{\frac{1}{2}}_{\mathbf Y}{\mathbf T}_{\mathbf X}{\mathbf B}\widetilde{\mathbf V}^{T}{\mathbf W}^{\frac{1}{2}}_{\mathbf Y} = \widetilde{\mathbf Z}_{\mathbf X} (\widetilde{\mathbf Z}_{\mathbf X}^{T}\widetilde{\mathbf Z}_{\mathbf X})^{+} \widetilde{\mathbf Z}_{\mathbf X}^T {\mathbf Z}_{\mathbf Y}.
\label{ols_equivalence}
\end{equation} This connection to OLS shows how to residualize (i.e.,
``regress out'' or ``correct for'') covariates, akin to how residuals
are computed in OLS. We do so with the original
\({\mathbf Z}_{\mathbf Y}\) as
\({\mathbf Z}_{\mathbf Y} - \widehat{\mathbf Z}_{\mathbf Y}\). PLS-CA-R
produces both a predicted and a residualized version of \({\mathbf Y}\).
Recall that
\(\widehat{{\mathbf Z}}_{\mathbf Y} = {\mathbf M}^{\frac{1}{2}}_{\mathbf Y}{\mathbf T}_{\mathbf X}{\mathbf B}\widetilde{\mathbf V}^{T}{\mathbf W}^{\frac{1}{2}}_{\mathbf Y}\).
We compute a reconstituted form of \({\mathbf Y}\) as \begin{equation}
\widehat{\mathbf Y} = (\widehat{{\mathbf Z}}_{\mathbf Y} + {\mathbf E}_{\mathbf Y}) \times ({\mathbf 1}^{T}{\mathbf Y}{\mathbf 1}),
\label{eq:Yhat}
\end{equation} which is the opposite steps of computing the deviations
matrix. We add back in the expected values and then scale the data by
the total sum of the original matrix. The same can be done for
residualized values (i.e., ``error'') as \begin{equation}
{\mathbf Y}_{\epsilon} = [({\mathbf Z}_{\mathbf Y} - \widehat{{\mathbf Z}}_{\mathbf Y}) + {\mathbf E}_{\mathbf Y}] \times ({\mathbf 1}^{T}{\mathbf Y}{\mathbf 1}).
\label{eq:Yresid}
\end{equation} Typically, \({\mathbf E}_{\mathbf Y}\) is derived the
data (as noted in Eq. \ref{eq:models}). However, the reconstituted space
could come from any model by way of generalized correspondence analysis
(GCA)
\citep{escofier1983analyse, escofier1984analyse, grassi1994correspondence, beaton2018generalization}.
With GCA we could use any reasonable alternates of
\({\mathbf E}_{\mathbf Y}\) as the model, which could be obtained from,
for examples, known priors, a theoretical model, out of sample data, or
population estimates. CA can then be applied directly to either
\(\widehat{\mathbf Y}\) or \({\mathbf Y}_{\epsilon}\). The same
reconstitution procedures can be applied fitted and residualized
versions of \({\mathbf X}\) as well.

\hypertarget{concluding-remarks}{%
\subsection{Concluding remarks}\label{concluding-remarks}}

Now that we have formalized PLS-CA-R, we want to point out small
variations, some caveats, and some additional features. We also briefly
discuss here---but show in the Appendix---how PLS-CA-R provides the
basis for numerous variations and broader generalizations.

In PLS-CA-R (and PLS-R) each subsequent \(\widetilde{\delta}\) is not
guaranteed to be smaller than the previous, with the exception that all
\(\widetilde \delta\) are smaller than the first. This is a by-product
of the iterative process and the deflation step. This problem poses two
issues: (1) visualization of component scores and (2) explained
variance. For visualization of the component scores---which use
\(\widetilde \delta\)---there is an alternative computation:
\({\mathbf F}^{'}_{J} = {\mathbf W}_{J}^{-1}\widetilde{\mathbf P}\) and
\({\mathbf F}^{'}_{K} = {\mathbf W}_{K}^{-1}\widetilde{\mathbf Q}\).
This alternative is referred to as ``asymmetric component scores'' in
the correspondence analysis literature
\citep{abdi2014correspondence, greenacre1993biplots}. Additionally,
instead of computing the variance per component or latent variable, we
can instead compute the amount of variance explained by each component
in \(\mathbf X\) and \(\mathbf Y\). To do so we require the sum of the
eigenvalues of each of the respective matrices per iteration via CA
(with the GSVD). Before the first iteration of PLS-CA-R we obtain the
full variance (i.e., the sum of the eigenvalues) of each matrix from
\(\mathrm{GSVD(} {\mathbf M}^{-1}_{\mathbf X}, {\mathbf Z}_{\mathbf X}, {\mathbf W}^{-1}_{\mathbf X} \mathrm{)}\)
and
\(\mathrm{GSVD(} {\mathbf M}^{-1}_{\mathbf Y}, {\mathbf Z}_{\mathbf Y}, {\mathbf W}^{-1}_{\mathbf Y} \mathrm{)}\),
which we respectively refer to as \({\phi}_{\bf X}\) and
\({\phi}_{\bf Y}\) We can compute the sum of the eigenvalues for each
deflated version of \({\mathbf Z}_{\mathbf X}\) and
\({\mathbf Z}_{\mathbf Y}\) through the GSVD just as above, referred to
as \({\phi}_{{\bf X},c}\) and \({\phi}_{{\bf Y},c}\). For each \(c\)
component the proportion of explained variance for each matrix is
\(\frac{{\phi}_{\bf X} - {\phi}_{{\bf X},c}}{{\phi}_{\bf X}}\) and
\(\frac{{\phi}_{\bf Y} - {\phi}_{{\bf Y},c}}{{\phi}_{\bf Y}}\).

In our formulation, the weights we use are derived from the \(\chi^2\)
assumption of independence. However nearly any choices of weights could
be used, so long as the weight matrices are at least positive
semi-definite (which requires the use of a generalized inverse). If
alternate row weights (i.e., \({\mathbf M}_{\mathbf X}\) or
\({\mathbf M}_{\mathbf Y}\)) were chosen, then the fitted values and
residuals are no longer guaranteed to be orthogonal (the same condition
is true in weighted OLS).

PLS-CA-R provides an important basis for various extensions. We can use
different PLS algorithms in addition to Algorithm \ref{algo:plscar}, for
examples, the PLS-SVD and canonical PLS algorithms. Our formulation of
PLS-CA-R provides the basis for, and leads to a more generalized
approach for other cross-decomposition techniques or optimizations
(e.g., canonical correlation, reduced rank regression). This basis also
allows for the use of alternate metrics as well as ridge-like
regularization. We show how PLS-CA-R leads to these generalizations in
the Appendix.

Though we formalized PLS-CA-R as a method for categorical (nominal) data
coded in complete disjunctive format (as seen in Table
\ref{table:disj}---see SEX columns---or Table
\ref{table:snps_models_disj}), PLS-CA-R can easily accomodate various
data types without loss of information. Specifically, both continuous
and ordinal data can be handled with relative ease and in a
``pseudo-disjunctive'' format, also referred to as ``fuzzy coding''
where complete disjunctive would be a ``crisp coding''
\citep{greenacrefuzzy}. We explain exactly how to handle various data
types as Section \ref{section:appex} progresses, which reflects more
``real world'' problems: complex, mixed data types, and multi-source
data.

\hypertarget{applications-examples}{%
\section{Applications \& Examples}\label{applications-examples}}

\label{section:appex}

In this section we provide examples with real data from the Alzheimer's
Disease Neuroimaging Initiative (ADNI). These examples illustrate how to
approach mixed data with PLS-CA-R and the multiple uses of PLS-CA-R
(e.g., for analyses, as a residualization procedure). We present three
sets of analyses. First we introduce of PLS-CA-R through a typical
example where we want predict genotypes (categorical) from groups
(categorical). Next we show how to predict genotypes from a small set of
behavioral and brain variables. This second example serves multiple
purposes: (1) how to recode and analyze mixed data (categorical,
ordinal, and continuous), (2) how to use PLS-CA-R, and (3) how to use
PLS-CA-R as residualization technique to remove effects from data prior
to subsequent analyses. Finally, we present a larger analysis with the
goal to predict genotypes from cortical uptake of AV45 (i.e., a
radiotracer) PET scan for beta-amyloid (``A\(\beta\)'') deposition. This
final example also makes use of residualization as illustrated in the
second example.

\hypertarget{adni-data}{%
\subsection{ADNI Data}\label{adni-data}}

\label{section:data}

Data used in the preparation of this article come from the ADNI database
(adni.loni.usc.edu). ADNI was launched in 2003 as a public-private
funding partnership and includes public funding by the National
Institute on Aging, the National Institute of Biomedical Imaging and
Bioengineering, and the Food and Drug Administration. The primary goal
of ADNI has been to test a wide variety of measures to assess the
progression of mild cognitive impairment and early Alzheimer's disease.
The ADNI project is the result of efforts of many coinvestigators from a
broad range of academic institutions and private corporations. Michael
W. Weiner (VA Medical Center, and University of California-San
Francisco) is the ADNI Principal Investigator. Subjects have been
recruited from over 50 sites across the United States and Canada (for
up-to-date information, see www.adni-info.org).

The data in the following examples come from several modalities from the
ADNI-GO/2 cohort. The data come from two sources available from the ADNI
download site (\url{http://adni.loni.usc.edu/}): genome-wide data and
the TADPOLE challenge data (\url{https://tadpole.grand-challenge.org/})
which contains a wide variety of data. Because the genetics data are
used in every example, we provide all genetics preprocessing details
here, and then describe any preprocessing for other data as we discuss
specific examples.

For all examples in this paper we use a candidate set of single
nucleotide polymorphisms (SNPs) extracted from the genome-wide data. We
extracted only SNPs associated with the \textit{MAPT}, \textit{APP},
\textit{ApoE}, and \textit{TOMM40} genes because these genes are
considered as candidate contributors to various AD pathologies:
\textit{MAPT} because it is associated with tau proteins, AD pathology,
or cognitive decline
\citep{myers_h1c_2005, trabzuni_mapt_2012, desikan_genetic_2015, cruchaga_rare_2012, peterson_variants_2014},
\textit{APP} because of its association with \(\beta\)-amyloid proteins
\citep{cruchaga_rare_2012, huang_apoe2_2017, jonsson_mutation_2012}, as
well as \textit{ApoE} and \textit{TOMM40} because of their strong
association with AD pathologies
\citep{linnertz_cis-regulatory_2014, roses_tomm40_2010-1, bennet_pleiotropy_2010, huang_apoe2_2017}.
SNPs were processed as follows via \citet{purcell2007plink} with
additional \texttt{R} code as necessary: minor allele frequency (MAF)
\(> 5\%\) and missingness for individuals and genotypes \(\leq 10\%\).
Because the SNPs are coded as categorical variables (i.e., for each
genotype) we performed an additional level of preprocessing: genotypes
\(> 5\%\) because even with MAF \(> 5\%\), it was possible that some
genotypes (e.g., the heterozygote or minor homozygote) could still have
very few occurrences. Therefore genotypes \(\leq 5\%\) were combined
with another genotype. In all cases the minor homozygote (`aa') fell
below that threshold and was then combined with its respective
heterozygote (`Aa'); thus some SNPs were effectively coded as the
dominant model (i.e., the major homozygote vs.~the presence of a minor
allele). See Table for an example of SNP data coding examples. From the
ADNI-GO/2 cohort there were 791 available participants with 134 total
SNPs across the four candidate genes. These 134 SNPs span 349 columns in
disjunctive coding (see Table \ref{table:snps_models_disj}). Other data
include diagnosis and demographics, some behavioral and cognitive
instruments, and several types of brain-based measures. We discuss these
additional data in further detail when we introduce these data.

\begin{table}[!h]

\caption{\label{tab:unnamed-chunk-2}\label{table:snps_models_disj} An example of a SNP with its genotypes for respective individuals and disjunctive coding for three types of genetic models: genotypic (three levels), dominant (two levels: major homozygote vs. presence of minor allele), and recessive (two levels: presence of major allele vs. minor homozygote).}
\centering
\begin{tabular}[t]{lllllllll}
\toprule
\multicolumn{1}{c}{ } & \multicolumn{1}{c}{SNP} & \multicolumn{3}{c}{Genotypic} & \multicolumn{2}{c}{Dominant} & \multicolumn{2}{c}{Recessive} \\
\cmidrule(l{2pt}r{2pt}){2-2} \cmidrule(l{2pt}r{2pt}){3-5} \cmidrule(l{2pt}r{2pt}){6-7} \cmidrule(l{2pt}r{2pt}){8-9}
  & Genotype & AA & Aa & aa & AA & Aa+aa & AA+Aa & aa\\
\midrule
SUBJ 1 & Aa & 0 & 1 & 0 & 0 & 1 & 1 & 0\\
SUBJ 2 & aa & 0 & 0 & 1 & 0 & 1 & 0 & 1\\
SUBJ 3 & aa & 0 & 0 & 1 & 0 & 1 & 0 & 1\\
SUBJ 4 & AA & 1 & 0 & 0 & 1 & 0 & 1 & 0\\
SUBJ 5 & Aa & 0 & 1 & 0 & 0 & 1 & 1 & 0\\
\addlinespace
SUBJ 6 & AA & 1 & 0 & 0 & 1 & 0 & 1 & 0\\
\bottomrule
\end{tabular}
\end{table}

\hypertarget{diagnosis-and-genotypes}{%
\subsection{Diagnosis and genotypes}\label{diagnosis-and-genotypes}}

\label{section:plscarda}

Our first example asks and answers the question: ``which genotypes are
associated with which diagnostic category?''. In ADNI, diagnosis at
baseline is a categorical variable that denotes which group each
participant belongs to (at the first visit): control (CN; \(N=\) 155),
subjective memory complaints (SMC; \(N=\) 99), early mild cognitive
impairment (EMCI; \(N=\) 277), late mild cognitive impairment (LMCI;
\(N=\) 134), and Alzheimer's disease (AD; \(N=\) 126). We present this
first example analysis in two ways: akin to a standard regression
problem a la Wold {[}\citet{wold_soft_1975};
\citet{wold_collinearity_1984}; \citet{wold_principal_1987}; cf.~Eq.
\ref{ols_equivalence}) and then again in the multivariate perspective of
``projection onto latent structures'' \citep{abdi_partial_2010-1}.

\begin{table}[!h]

\caption{\label{tab:sample_descriptives}\label{table:desctab} Descriptives and demographics for the sample. AD = Alzheimer's Disease, CN = control, EMCI = early mild cognitive impairment, LMCI = late mild cognitive impairment, SMC = subjective memory complaints.}
\centering
\begin{tabular}[t]{lllll}
\toprule
  & N & AGE mean (sd) & EDU mean (sd) & Males (Females)\\
\midrule
AD & 126 & 74.53 (8.42) & 15.78 (2.68) & 75 (51)\\
CN & 155 & 74 (6.02) & 16.41 (2.52) & 80 (75)\\
EMCI & 277 & 71.14 (7.39) & 15.93 (2.64) & 156 (121)\\
LMCI & 134 & 72.24 (7.68) & 16.43 (2.57) & 73 (61)\\
SMC & 99 & 72.19 (5.71) & 16.81 (2.51) & 40 (59)\\
\bottomrule
\end{tabular}
\end{table}

For this example we refer to diagnosis groups as the predictors
(\({\bf X}\)) and the genotypic data as the responses (\({\bf Y}\)).
Both data types are coded in disjunctive format (see Tables
\ref{table:disj} and \ref{table:snps_models_disj}). Because there are
five columns (groups) in \({\bf X}\), PLS-CA-R produces only four latent
variables (a.k.a. components). Table \ref{table:r2ex1} presents the
cumulative explained variance for both \({\bf X}\) and \({\bf Y}\) and
shows that groups explain only a small amount of genotypic variance:
\(R^2=\) 0.0065.

\begin{table}[!h]

\caption{\label{tab:unnamed-chunk-3}\label{table:r2ex1} The R-squared values over the four latent variables for both groups and genotypes. The full variance of groups is explained over the four latent variables. The groups explained 0.65\% of the genotypes.}
\centering
\begin{tabular}[t]{lrr}
\toprule
  & X (groups) R-squared cumulative & Y (genotypes) R-squared cumulative\\
\midrule
Latent variable 1 & 0.25 & 0.0026\\
Latent variable 2 & 0.50 & 0.0042\\
Latent variable 3 & 0.75 & 0.0055\\
Latent variable 4 & 1.00 & 0.0065\\
\bottomrule
\end{tabular}
\end{table}

In a simple regression-like framework we can compute the variance
contributed by genotypes or group (i.e., levels of variables) or
variance contributed by entire variables (in this example: SNPs). First
we compute the contributions to the variance of the genotypes as the sum
of the squared loadings for each item:
\([(\widetilde{\bf V} \odot \widetilde{\bf V}){\bf 1}] \times C^{-1}\),
where \({\bf 1}\) is a conformable vector of ones. Total contribution
takes values between 0 and 1 and describe the proportion of variance for
each genotype. Because the contributions are squared loadings, they are
additive and so we can compute the contributions for a SNP. A simple
criterion to identify genotypes or SNPs that contribute to the model is
to identify which genotype or SNP contributes more variance than
expected, which is one divided by the total number of original variables
(i.e., SNPs). This criterion can be applied on the whole or
component-wise. We show the genotypes and SNPs with above expected
variance for the whole model (i.e., high contributing variables a
regression framework) in Figure \ref{fig:leverages_ex1}.

\begin{figure}[!hbtp]

{\centering \includegraphics[width=.8\textwidth,height=.8\textheight]{PLSCAR_to_a_GPLS_files/figure-latex/unnamed-chunk-4-1} 

}

\caption{\label{fig:leverages_ex1} Regression approach to prediction of genotypes from groups. Contributions across all components for genotypes (A; top) and the SNPs (B; bottom) computed as the summation of genotypes within a SNP. The horizontal line shows the expected variance and we only highlight genotypes (A; top) or SNPs (B; bottom) greater than the expected variance. Some of the highest contributing genotypes (e.g., AA and AG genotypes for rs769449) or SNPs (e.g.,  rs769449 and rs20756560) come from the APOE and TOMM40 genes.}\label{fig:unnamed-chunk-4}
\end{figure}

Though PLS-R was initally developed as a regression
approach---especially to handle collinear predictors \citep[see
explanations in][]{wold_collinearity_1984}---it is far more common to
use PLS to find latent structures (i.e., components or latent variables)
\citep{abdi_partial_2010-1}. From here on we show the latent variable
scores (observations) and component scores (variables) for the first two
latent variables/components in Figure \ref{fig:contributions_ex1}. The
first latent variable scores (Fig. \ref{fig:contributions_ex1}a) shows a
gradient from the control (CN) group through to the Alzheimer's Disease
(AD) groups (CN to SMC to EMCI to LMCI to AD). The second latent
variable shows a dissociation of the EMCI group from all other groups
(Fig. \ref{fig:contributions_ex1}b). Figure \ref{fig:contributions_ex1}c
and d show the component scores for the variables. Genotypes on the left
side of first latent variable (horizontal axis in Figs.
\ref{fig:contributions_ex1}c and d) are more associated with CN and SMC
than the other groups, whereas genotypes on the right side are more
associated with AD and LMCI than with the other groups. Through the
latent structures approach we can more clearly see the relationships
between groups and genotypes. Because we treat the data categorically
and code for genotypes, we can identify the specific genotypes that
contribute to these effects. For example the `AA' genotype of rs769449
and the `GG' genotype of rs2075650 are more associated with AD and LMCI
than with the other groups. In contrast, the `TT' genotype of rs405697
and the `TT' genotype rs439401 are more associated with the CN group
than other groups (and thus could suggest potential protective effects).

\begin{figure}[!hbtp]

{\centering \includegraphics[width=.8\textwidth,height=.8\textheight]{PLSCAR_to_a_GPLS_files/figure-latex/unnamed-chunk-5-1} 

}

\caption{\label{fig:contributions_ex1} Latent variable projection approach to prediction of genotypes from groups. (A) and (B) show the latent variable scores for latent variables (LVs; components) one and two, respectively; (C) shows the component scores of the groups, and (D) shows the component scores of the genotypes. In (D) we highlight genotypes with above expected contribution to Latent Variable (Component) 1 in purple and make all other genotypes gray.}\label{fig:unnamed-chunk-5}
\end{figure}

\begin{figure}[!hbtp]

{\centering \includegraphics[width=.8\textwidth,height=.8\textheight]{PLSCAR_to_a_GPLS_files/figure-latex/unnamed-chunk-6-1} 

}

\caption{\label{fig:discriminant_ex1} Discriminant PLS-CA-R. (A) shows the component scores for the group on Latent Variables (LV) 1 and 2 (horizontal and vertical respectively), (B) shows the latent variable scores for the genotype ('LY') LV scores for LVs 1 and 2, colored by \textit{a priori} group association, (C) shows the latent variable scores for the genotype ('LY') LV scores for LVs 1 and 2, colored by \textit{assigned} group association (i.e., nearest group assignment across all LVs), and (D) shows correct vs. incorrect assignment in black and gray, respectively.}\label{fig:unnamed-chunk-6}
\end{figure}

This group-based analysis is also a discriminant analysis because it
maximally separates groups. Thus we can classify observations by
assigning them to the closest group. To correctly project observations
onto the latent variables we compute
\({\bf M}_{\bf Y}^{-\frac{1}{2}}{\mathbf L}_{\mathbf Y} = [{\bf O}_{\bf Y} \oslash ({\bf m}_{\bf Y}{\bf 1}^T)]{\bf F}_{K}\boldsymbol{\Delta}^{-1}\)
where \(1\) is a \(1 \times K\) vector of ones where
\({\bf O}_{\bf Y} \oslash ({\bf m}_{\bf Y}{\bf 1}^T)\) are ``row
profiles'' of \({\bf Y}\) (i.e., each element of \({\bf Y}\) divided by
its respective row sum). Observations from
\({\bf M}_{\bf Y}^{-\frac{1}{2}}{\mathbf L}_{\mathbf Y}\) are then
assigned to the closest group in \({\mathbf F}_{J}\), either for per
component, across a subset of components, or all components. For this
example we use the full set (four) of components. The assigned groups
can then be compared to the \emph{a priori} groups to compute a
classification accuracy. Figure \ref{fig:discriminant_ex1} shows the
results of the discriminant analysis but only visualized on the first
two components. Figures \ref{fig:discriminant_ex1}a and b respectively
show the scores for \({\mathbf F}_{J}\) and
\({\bf M}_{\bf Y}^{-\frac{1}{2}}{\mathbf L}_{\mathbf Y}\). Figure
\ref{fig:discriminant_ex1}c shows the assignment of observations to
their closest group. Figure \ref{fig:discriminant_ex1}d visualizes the
accuracy of the assignment, where observations in black are correct
assignments (gray are incorrect assignments). The total classification
accuracy 38.69\% (where chance accuracy was 23.08\%). Finally, typical
PLS-R discriminant analyses are applied in scenarios where a small set
of, or even a single, (typically) categorical responses are predicted
from many predictors \citep{perez-enciso_prediction_2003}. However, such
an approach appears to be ``over optimistic'' in its prediction and
classification \citep{rodriguez-perez_overoptimism_2018}, which is why
we present discriminant PLS-CA-R more akin to a typical regression
problem (i.e., here a single predictor with multiple responses).

\begin{table}[!h]

\caption{\label{tab:unnamed-chunk-7}\label{table:assign_ex1} The \textit{a priori} (rows) vs. assigned (columns) accuracies for the discriminant analysis. AD = Alzheimer's Disease, CN = control, EMCI = early mild cognitive impairment, LMCI = late mild cognitive impairment, SMC = subjective memory complaints.}
\centering
\begin{tabular}[t]{lrrrrr}
\toprule
\multicolumn{1}{c}{\em  } & \multicolumn{5}{c}{\em Assigned} \\
\cmidrule(l{2pt}r{2pt}){2-6}
  & CN & SMC & EMCI & LMCI & AD\\
\midrule
CN & 62 & 15 & 54 & 16 & 17\\
SMC & 20 & 40 & 33 & 18 & 10\\
EMCI & 34 & 20 & 110 & 23 & 29\\
LMCI & 14 & 12 & 39 & 44 & 20\\
AD & 25 & 12 & 41 & 33 & 50\\
\bottomrule
\end{tabular}
\end{table}

\hypertarget{mixed-data-and-residualization}{%
\subsection{Mixed data and
residualization}\label{mixed-data-and-residualization}}

\label{section:mixed}

Our second example illustrates the prediction of genotypes from brain
and behavioral variables: (1) three behavioral/clinical scales: Montreal
Cognitive Assessment (MoCA) \citep{nasreddine_montreal_2005}, Clinical
Dementia Rating-Sum of Boxes (CDRSB) \citep{morris1993clinical}, and
Alzheimer's Disease Assessment Scale (ADAS13)
\citep{skinner_alzheimers_2012}, (2) volumetric brain measures in
\(\textrm{mm}^3\): hippocampus (HIPPO), ventricles (VENT), and whole
brain (WB), and (3) global estimates of brain function via PET scans:
average FDG (for cerebral blood flow; metabolism) in angular, temporal,
and posterior cingulate and average AV45 (A\(\beta\) tracer) standard
uptake value ratio (SUVR) in frontal, anterior cingulate, precuneus, and
parietal cortex relative to the cerebellum. This example higlights two
features of PLS-CA-R: (1) the ability to accomodate mixed data types
(continuous, ordinal, and categorical) and (2) as a way to residualize
(orthogonalize; cf.~Eq. \ref{eq:Yresid}) with respect to known or
assumed covariates.

Here, the predictors encompass a variety of data types: all of the brain
markers (volumetric MRI estimates, functional PET estimates) and the
ADAS13 are quantitative variables, whereas the MoCA and the CDRSB can be
considered as ordinal data. Continuous and ordinal data types can be
coded into what is called thermometer \citep{beaton2018generalization},
fuzzy, or ``bipolar'' coding (because it has two poles)
\citep{greenacrefuzzy}; an idea initially propsosed by Escofier for
continuous data \citep{escofier_traitement_1979}. The ``Escofier
transform'' allows continuous data to be analyzed by CA and produces the
exact same results as PCA \citep{escofier_traitement_1979}. The same
principles can be applied to ordinal data as well
\citep{beaton2018generalization}. Continuous and ordinal data can be
transformed into a ``pseudo-disjunctive'' format that behaves exactly
like complete disjunctive data (see Table \ref{table:disj}) but
preserves the values (as opposed to binning, or dichotomizing). Here, we
refer to the transform for continuous data as the ``Escofier transform''
or ``Escofier coding'' \citep{beaton_partial_2016} and the transform for
ordinal data as the ``thermometer transform'' or ``thermometer coding''.
Because continuous, ordinal, and categorical data can all be transformed
into a disjunctive-like format, they can all be analyzed with PLS-CA-R.

The next example identifies the relationship between brain and
behavioral markers of AD and genetics. We use the brain and behavioral
data as the predictors and the genetics as responses. However, both data
sets have their own covariates (some of which are confounds): age, sex,
and education influence the behavioral and brain data, where sex, and
population origins influence the genotypic variables. For genetic
covariates, we use proxies of population origins: self-identified race
and ethnicity categories. Both sets of covariates are miixtures of data
types (e.g., sex is categorical, age is generally continuous). So, in
this example, we illustrate the mixed analysis in two ways---unadjusted
and then adjusted for these covariates. First we show the effects of the
covariates on the separate data sets, and then compare and contrast
adjusted vs.~unadjusted analyes. For these analyses, the volumetric
brain data were also normalized (divided by) by intracranial volume
prior to these analyses to create proportions of total volume for each
brain structure.

First we show the PLS-CA-R between each data set and their respective
covariates The main effects of age, sex, and education explained 11.17\%
of the variance of the behavioral and brain data, where the main effects
of sex, race, and ethnicity explained 2.1\% of the variance of the
genotypic data. The first two components of each analysis are shown in
Figure \ref{fig:confound_predictors_ex2}. In the brain and behavioral
data, age explains a substantial amount of variance for Component 1. In
the genotypic analysis, self-identified race and ethnicity are the
primary overall effects, where the first two components are explained by
those that identify as black or African-American (Component 1) vs.~those
that identify as Asian, Native, Hawaiian, and/or Latino/Hispanic
(Component 2). Both data sets were reconstituted (i.e.,
\({\mathbf Y}_{\epsilon}\) from Eq. \ref{eq:Yresid}) from their
residuals.

Next we performed two analyses with the same goal: to understand the
relationship between genetics and behavioral and brain markers of AD. In
the unadjusted analysis, the brain and behavioral data explained 1.6\%
of variance in the genotypic data, whereas in the adjusted analysis, the
brain and behavioral data explained 1.54\% of variance in the genotypic
data. The first two components of the PLS-CA-R results can be seen in
Figure \ref{fig:confound_predictors_ex2}.

\begin{figure}[!hbtp]

{\centering \includegraphics[width=.8\textwidth,height=.8\textheight]{PLSCAR_to_a_GPLS_files/figure-latex/unnamed-chunk-9-1} 

}

\caption{\label{fig:confound_predictors_ex2} PLS-CA-R used as a way to residualize (orthogonalize) data. The top figures (A) and (B) show prediction of the brain and behavior markers from age, sex, and education. Gray items are one side (lower end) of the "bipolar" or pseudo-disjunctive variables. The bottom figures (C) and (D) show the prediction of genotypes from sex, race, and ethnicity.}\label{fig:unnamed-chunk-9}
\end{figure}

In the unadjusted analysis (Figures \ref{fig:confound_predictors_ex2}a
and c) vs.~the adjusted analysis (Figures
\ref{fig:confound_predictors_ex2}b and d), we can see some similarities
and differences, especially with respect to the behavioral and brain
data. AV45 shows little change after residualization, and generally
explains a substantial amount of variance as it contributes highly to
the first two components in both analyses. The effects of the structural
data---especially the hippocampus---are dampened after adjustment (see
Figures \ref{fig:confound_predictors_ex2}a vs b), where the effects of
FDG and CDRSB are now (relatively) increased (see Figures
\ref{fig:confound_predictors_ex2}a vs b). On the subject level, the
differences are small but noticeable, especially for distinguishing
between groups (see Figure \ref{fig:lv_compare_ex2}). One important
effect is that on a spectrum from CON to AD, we can see that the
residualization has a larger impact on the CON side, where the AD side
remains somewhat homgeneous (see Figure \ref{fig:lv_compare_ex2}c) for
the brain and behavioral variables. With respect to the genotypic LV,
there is much less of an effect (see Figure \ref{fig:lv_compare_ex2}d),
wherein the observations appear relatively unchanged. However, both pre-
(horizontal axis; Figure \ref{fig:lv_compare_ex2}d) and post- (vertical
axis; Figure \ref{fig:lv_compare_ex2}d) residualization shows that there
are individuals with unique genotypic patterns that remain unaffected by
the residualization process (i.e., those at the tails).

From this point forward we emphasize the results from the adjusted
analyses because they are more realistic in terms of how analyses are
performed. For this we refer to Figure \ref{fig:lv_compare_ex2}b---which
shows the latent variable scores for the observations and the averages
of those scores for the groups---and Figures
\ref{fig:original_residualized_ex2}b and
\ref{fig:original_residualized_ex2}d---which show, respectively, the
component scores for the brain and behavioral markers and the component
scores for the genotypes. The first latent variable (Fig.
\ref{fig:lv_compare_ex2}b) shows a gradient from control (CON) on the
left to Alzheimer's Disease (AD) on the right. Brain and behavioral
variables on the right side of the first component (horizontal axis in
Fig. \ref{fig:original_residualized_ex2}b) are more associated with
genotypes on the right side (Fig. \ref{fig:original_residualized_ex2}d),
where brain and behavioral variables on the left side of the horizontal
axis are more associated with genotypes on the left side. In particular,
the AA genotype of rs769449, GG genotype of rs2075650, GG genotype of
rs4420638, and AA genotype of rs157582 (amongst others) are related to
increased AV45 (AV45+), decreased FDG (FDG-), and increased ADAS13
scores (ADAS13+), whereas the TT genotype of rs405697, GG genotype of
rs157580, and TC+TT genotypes of rs7412 (amongst others) are more
associated with control or possibly protective effects (i.e., decreased
AV4, increased FDG, and decreased ADAS13 scores).

\begin{figure}[!hbtp]

{\centering \includegraphics[width=.8\textwidth,height=.8\textheight]{PLSCAR_to_a_GPLS_files/figure-latex/unnamed-chunk-10-1} 

}

\caption{\label{fig:original_residualized_ex2} PLS-CA-R to predict genotypes from brain and behavioral markers on the original and residualized data shown on the first two latent variables (components). The top figures (A) and (B) show the component scores for the brain and behavioral markers for the original and residualized data, respectively, and the bottom figures (C) and (D) show the component scores for the genotypes for the original and residualized data, respectively.}\label{fig:unnamed-chunk-10}
\end{figure}

\begin{figure}[!hbtp]

{\centering \includegraphics[width=.8\textwidth,height=.8\textheight]{PLSCAR_to_a_GPLS_files/figure-latex/unnamed-chunk-11-1} 

}

\caption{\label{fig:lv_compare_ex2} Latent variable scores (observations) for the first latent variable. The top figures (A) and (B) show the projection of the latent variable scores from each set: LX are the brain and behavioral markers, where as LY are the genotypes, for the original and residualized, respectively. The bottom figures (C) and (D) show the the original and residualized scores for the first latent variable compared to one another for each set: the brain and behavioral markers (LX) and the genotypes (LY), respectively.}\label{fig:unnamed-chunk-11}
\end{figure}

\hypertarget{suvr-and-genotypes}{%
\subsection{SUVR and genotypes}\label{suvr-and-genotypes}}

\label{section:big}

In this final example we use all of the features of PLS-CA-R: mixed data
types within and between data sets, each with covariates (and thus
require residualization). The goal of this example is to predict
genotypes from \(\beta-\)amyloid burden (``AV45 uptake'') across regions
of the cortex. In this case, because the AV45 uptake data are are
strictly non-negative, we effectively treat these data as counts; that
is, we would normally apply CA directly to this one data table. Of
course, this is only one possible way to handle such data. It is
possible to treat these data as row-wise proportions (i.e., percentage
of total uptake per region within each subject) or even as continuous
data. Ultimately, it is up to analysts and experts to decide how to best
treat such data, and how it fits into the analysis framework.

Because not all subjects have complete AV45 and genotypic data, the
sample for this example is slightly smaller: \(N=\) 778. Ethnicity,
race, and sex (all categorical) explains 2.07\% of the variance in the
genotypic data where age (numeric), education (ordinal), and sex
(categorical) explains 2.22\% of the variance in the in the AV45 uptake
data. Overall, AV45 brain data explains 9.08\% of the variance in the
genotypic data. With the adjusted data we can now perform our intended
analyses. Although this analysis produced 67 components (latent
variables), we focus on just the first (0.57\% of genotypic variance
explained by AV45 brain data).

\begin{figure}[!hbtp]

{\centering \includegraphics[width=.8\textwidth,height=.8\textheight]{PLSCAR_to_a_GPLS_files/figure-latex/unnamed-chunk-13-1} 

}

\caption{\label{fig:brain_genotypes_ex2} PLS-CA-R to predict genotypes from amyloid burden ("AV45 uptake"). The top figure (A) shows the latent variable scores for the observations on the first latent variable with group averages. The bottom figures (B) and (C) show the amyloid burden in cortical regions and the genotypes, respecively. In (A) we see a gradient from the Alzheimer's Disease (AD) group to the control (CON) group. Items above expected contribution to variance on the first LV are in purple.}\label{fig:unnamed-chunk-13}
\end{figure}

The first latent variable in Figure \ref{fig:brain_genotypes_ex2}a is
associated with only the horizontal axes (Component 1) in Figure
\ref{fig:brain_genotypes_ex2}b and c.~The horizontal axis in Fig.
\ref{fig:brain_genotypes_ex2}a is associated with the horizontal axis in
Fig. \ref{fig:brain_genotypes_ex2}b whereas the vertical axis in Fig.
\ref{fig:brain_genotypes_ex2}a is associated with the horizontal axis in
Fig. \ref{fig:brain_genotypes_ex2}c.~The first latent variable (Figure
\ref{fig:brain_genotypes_ex2}a) shows a gradient: from left to right we
see the groups configured from CN to AD. On the first latent variable we
do also see a group-level dissociation where AD+LMCI are entirely on one
side whereas EMCI+SMC+CN are on the opposite side for both
\({\bf L}_{\bf X}\) (AV45 uptake, horizontal) and \({\bf L}_{\bf Y}\)
(genotypes, vertical); effectively the means of AD and LMCI exist in the
upper right quadrant and the means of the EMCI, SMC, and CN groups exist
in the lower left quadrant. Higher relative AV45 uptake for the regions
on the left side of Component 1 are more associated with EMCI, SMC, and
CN than with the other groups, whereas higher relative AV45 uptake for
the regions on the right side of Component 1 are more associated with AD
and LMCI (Fig. \ref{fig:brain_genotypes_ex2}b). The genotypes on the
left side are associated with the uptake in regions on the left side;
likewise the genotypes on the right side are associated with the uptake
in regions on the right side (Fig. \ref{fig:brain_genotypes_ex2}c). For
example, LV/Component 1 shows relative uptake in right and left frontal
pole, rostral middle frontal, and medial orbitofrontal regions are more
associated with the following genotypes: AA and AG from rs769449, GG
from rs2075650, GG from rs4420638, and AA from rs157582, than with other
genotypes; these effects are generally driven by the AD and LMCI groups.
Conversely, LV/Component 1 shows higher relative uptake in right and
left lingual, cuneus, as well left parahippocampal and left entorhinal
are more associated with the following genotypes: TT from rs405697, GG
from rs6859, TC+TT from rs7412, TT from rs2830076, GG from rs157580, and
AA from rs4420638 genotypes than with other genotypes; these effects are
generally driven by the CN, SMC, and EMCI cohorts. In summary, the
PLS-CA-R results show that particular patterns of regional AV45 uptake
predict particular genotypic patterns across many SNPs, and that the
sources these effects are generally driven by the groups. Furthermore
the underlying brain and genotypic effects of the groups exist along a
spectrum of severity.

\hypertarget{discussion}{%
\section{Discussion}\label{discussion}}

\label{section:Disc}

Many modern studies, like ADNI, measure individuals with a variety of
scales such as: genetics and genomics, brain structure and function,
many aspects of cognition and behavior, batteries of clinical measures,
and so on. These data are complex, heterogeneous, and more frequently
these data are ``wide'' (many more variables than subjects) instead of
``big'' (more subjects than variables). But many current strategies and
approaches to handle such multivariate heterogeneous data often requires
compromises or sacrifices (e.g., the presumption of single numeric model
for categorical data such as the additive model for SNPs; z-scores of
ordinal values; or ``dichotomania''
(\url{https://www.fharrell.com/post/errmed/\#catg}): the binning of
continuous values into categories). Many of these strategies and
approaches presume data are interval scale. Because of the many features
and flexibility of PLS-CA-R---e.g., best fit to predictors, orthogonal
latent variables, accommodation for virtually any data type---we are
able to identify distinct variables and levels (e.g., genotypes) that
define or contribute to control (CON) vs.~disease (AD) effects (e.g.,
Fig. \ref{fig:contributions_ex1}) or reveal particular patterns anchored
by the polar control and disease effects (CON \(\rightarrow\) SMC
\(\rightarrow\) EMCI \(\rightarrow\) LMCI \(\rightarrow\) AD; see, e.g.,
Fig. \ref{fig:brain_genotypes_ex2}).

While we focused on particular ways of coding and transforming data,
there are many alternatives that could be used with PLS-CA-R. For
example, we used a disjunctive approach for SNPs because they are
categorical, which matches the genotypic model. However, through various
disjunctive schemes, or other forms of Escofier or fuzzy coding, we
could have used any genetic model: if all SNPs were coded as the major
vs.~the minor allele (`AA' vs.~\{`Aa+aa'\}), this would be the dominant
model, or we could have assumed the additive model ---i.e., 0, 1, 2 for
`AA', `Aa', and `aa', respectively---and transformed the data with the
ordinal approach (but we strongly emphasize \emph{not} the continuous
approach). We previously provided a comprehensive guide on how to
transform various SNP genetic models for use in PLS-CA and CA elsewhere
\citep[see Appendix of][]{beaton_partial_2016}. Furthermore, we only
highlighted one of many possible methods to transform ordinal data. The
term ``fuzzy coding'' applies more generally to the recoding of ordinal,
ranked, preference, and even continuous data. The ``fuzzy coding''
approach is itself fuzzy, and includes a variety of transformation
schemes, all of which conform to the same properties as disjunctive
data. The many ``fuzzy'' and ``doubling'' coding schemes are generally
found in \citet{escofier_traitement_1979},
\citet{lebart_multivariate_1984}, or \citet{greenacrefuzzy}. However,
for ordinal data---especially with fewer than or equal to 10 levels, and
without excessively rare (\(\leq 1\)\%) occurences---we recommend to
treat ordinal values as categorical levels. When ordinal data are
treated as categorial (and disjunctively coded), greater detail about
the levels emerges and in most cases reveal non-linear patterns of the
ordinal levels.

We introduced PLS-CA-R in a way that emphasizes various recoding schemes
to accomodate different data types. We designed PLS-CA-R as a mixed-data
generalization of PLSR, which provides one strategy to identify latent
variables and perform regression when standard assumptions are not met
(e.g., HDLSS or high collinearity). PLS-CA-R---and GPLS---addresses the
need of many fields that require \textit{data type general} methods
across multi-source and multi-domain data sets where we require careful
considerations about how we prepare and understand our data
\citep{nguyen2019ten}. We introduced PLS-CA-R in a way that emphasizes
various recoding schemes to accomodate different data types all with
respect to CA and the \(\chi^2\) model. PLS-CA-R provides key features
necessary for data analyses as data-rich and data-heavy disciplines and
fields rapidly move towards and depend on fundamental techniques in
machine and statistical learning (e.g., PLSR, CCA). Finally, with
techniques such as mixed-data MFA \citep{becue-bertaut_multiple_2008},
PLS-CA-R provides a much needed basis for development of future methods
designed for such complex data sets.

Finally, we want to remind the reader to see the Appendix. In the
Appendix, we show (1) how PLS-CA-R generalizes most cross-decomposition
techniques, (2) alternate PLS algorithms, (3) suggestions on alternate
metrics, and finally (4) a ridge-like regularization approach that
applies to PLS-CA-R and all the techniques that PLS-CA-R generalizes.

\hypertarget{acknowledgements}{%
\subsection{Acknowledgements}\label{acknowledgements}}

Data collection and sharing for this project was funded by the
Alzheimer's Disease Neuroimaging Initiative (ADNI) (National Institutes
of Health Grant U01 AG024904) and DOD ADNI (Department of Defense award
number W81XWH-12-2-0012). ADNI is funded by the National Institute on
Aging, the National Institute of Biomedical Imaging and Bioengineering,
and through generous contributions from the following: AbbVie,
Alzheimer's Association; Alzheimer's Drug Discovery Foundation; Araclon
Biotech; BioClinica, Inc.; Biogen; Bristol-Myers Squibb Company;
CereSpir, Inc.; Cogstate; Eisai Inc.; Elan Pharmaceuticals, Inc.; Eli
Lilly and Company; EuroImmun; F. Hoffmann-La Roche Ltd and its
affiliated company Genentech, Inc.; Fujirebio; GE Healthcare; IXICO
Ltd.; Janssen Alzheimer Immunotherapy Research \& Development, LLC.;
Johnson \& Johnson Pharmaceutical Research \& Development LLC.;
Lumosity; Lundbeck; Merck \& Co., Inc.; Meso Scale Diagnostics, LLC.;
NeuroRx Research; Neurotrack Technologies; Novartis Pharmaceuticals
Corporation; Pfizer Inc.; Piramal Imaging; Servier; Takeda
Pharmaceutical Company; and Transition Therapeutics. The Canadian
Institutes of Health Research is providing funds to support ADNI
clinical sites in Canada. Private sector contributions are facilitated
by the Foundation for the National Institutes of Health (www.fnih.org).
The grantee organization is the Northern California Institute for
Research and Education, and the study is coordinated by the Alzheimer's
Therapeutic Research Institute at the University of Southern California.
ADNI data are disseminated by the Laboratory for Neuro Imaging at the
University of Southern California.

\bibliographystyle{agsm}
\bibliography{plscar.bib}

\end{document}
